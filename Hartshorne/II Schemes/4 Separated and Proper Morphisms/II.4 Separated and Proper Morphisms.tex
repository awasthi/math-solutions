\documentclass[12pt,letterpaper]{article}
\usepackage{geometry}
\geometry{letterpaper}
\usepackage{microtype}
\usepackage{amsmath,amssymb,amsthm,mathrsfs}
\usepackage{mathtools}
\usepackage{ifpdf}
  \ifpdf
    \setlength{\pdfpagewidth}{8.5in}
    \setlength{\pdfpageheight}{11in}
  \else
\fi
\usepackage{hyperref}

\usepackage{tikz}
\usepackage{tikz-cd}
\usetikzlibrary{decorations.markings}
\tikzset{
  open/.style = {decoration = {markings, mark = at position 0.5 with { \node[transform shape] {\tikz\draw[fill=white] (0,0) circle (.3ex);}; } }, postaction = {decorate} },
  closed/.style = {decoration = {markings, mark = at position 0.5 with { \node[transform shape, xscale = .8, yscale=.4] {\upshape{/}}; } }, postaction = {decorate} },
  imm/.style = {decoration = {markings, mark = at position 0.3 with { \node[transform shape, xscale = .8, yscale=.4] {\upshape{/}}; }, mark = at position 0.6 with { \node[transform shape] {\tikz\draw[fill=white] (0,0) circle (.3ex);}; } }, postaction = {decorate} }
}

\usepackage{braket}

\usepackage[utf8]{inputenc}
\usepackage{csquotes}
\usepackage[american]{babel}
\usepackage[style=alphabetic,firstinits=true,backend=biber,texencoding=utf8,bibencoding=utf8]{biblatex}
\bibliography{../Hartshorne}
\AtEveryBibitem{\clearfield{url}}
\AtEveryBibitem{\clearfield{doi}}
\AtEveryBibitem{\clearfield{issn}}
\AtEveryBibitem{\clearfield{isbn}}
\renewbibmacro{in:}{}
\DeclareFieldFormat{postnote}{#1}
\DeclareFieldFormat{multipostnote}{#1}

\renewcommand{\theenumi}{$(\alph{enumi})$}
\renewcommand{\labelenumi}{\theenumi}

\newcounter{enumacounter}
\newenvironment{enuma}
{\begin{list}{$(\alph{enumacounter})$}{\usecounter{enumacounter} \parsep=0em \itemsep=0em \leftmargin=2.75em \labelwidth=1.5em \topsep=0em}}
{\end{list}}
\newcounter{enumdcounter}
\newenvironment{enumd}
{\begin{list}{$(\arabic{enumdcounter})$}{\usecounter{enumdcounter} \parsep=0em \itemsep=0em \leftmargin=1.75em \labelwidth=1.5em \topsep=0em}}
{\end{list}}
\newtheorem*{theorem}{Theorem}
\newtheorem*{universalproperty}{Universal Property}
\newtheorem{problem}{Exercise}[section]
\newtheorem{subproblem}{Exercise}[problem]
\newtheorem{lemma}{Lemma}%[section]
\newtheorem{proposition}{Proposition}
\newtheorem{property}{Property}[problem]
\newtheorem*{lemma*}{Lemma}
\theoremstyle{definition}
\newtheorem*{definition}{Definition}
\newtheorem*{claim}{Claim}
\theoremstyle{remark}
\newtheorem*{remark}{Remark}

\numberwithin{equation}{section}
\numberwithin{figure}{problem}
\renewcommand{\theequation}{\arabic{section}.\arabic{equation}}

\DeclareMathOperator{\Ann}{Ann}
\DeclareMathOperator{\Ass}{Ass}
\DeclareMathOperator{\Supp}{Supp}
\DeclareMathOperator{\WeakAss}{\widetilde{Ass}}
\let\Im\relax
\DeclareMathOperator{\Im}{im}
\DeclareMathOperator{\Spec}{Spec}
\DeclareMathOperator{\SPEC}{\mathbf{Spec}}
\DeclareMathOperator{\Sp}{sp}
\DeclareMathOperator{\maxSpec}{maxSpec}
\DeclareMathOperator{\Hom}{Hom}
\DeclareMathOperator{\Soc}{Soc}
\DeclareMathOperator{\Ht}{ht}
\DeclareMathOperator{\A}{\mathcal{A}}
\DeclareMathOperator{\V}{\mathbf{V}}
\DeclareMathOperator{\Aut}{Aut}
\DeclareMathOperator{\Char}{char}
\DeclareMathOperator{\Frac}{Frac}
\DeclareMathOperator{\Proj}{Proj}
\DeclareMathOperator{\stimes}{\text{\footnotesize\textcircled{s}}}
\DeclareMathOperator{\End}{End}
\DeclareMathOperator{\Ker}{Ker}
\DeclareMathOperator{\Coker}{coker}
\DeclareMathOperator{\LCM}{LCM}
\DeclareMathOperator{\Div}{Div}
\DeclareMathOperator{\id}{id}
\DeclareMathOperator{\Cl}{Cl}
\DeclareMathOperator{\dv}{div}
\DeclareMathOperator{\Gr}{Gr}
\DeclareMathOperator{\pr}{pr}
\DeclareMathOperator{\trd}{tr.d.}
\DeclareMathOperator{\rank}{rank}
\DeclareMathOperator{\codim}{codim}
\DeclareMathOperator{\sgn}{sgn}
\DeclareMathOperator{\GL}{GL}
\newcommand{\GR}{\mathbb{G}\mathrm{r}}
\newcommand{\gR}{\mathrm{Gr}}
\newcommand{\EE}{\mathscr{E}}
\newcommand{\FF}{\mathscr{F}}
\newcommand{\GG}{\mathscr{G}}
\newcommand{\HH}{\mathscr{H}}
\newcommand{\II}{\mathscr{I}}
\newcommand{\LL}{\mathscr{L}}
\newcommand{\MM}{\mathscr{M}}
\newcommand{\OO}{\mathcal{O}}
\newcommand{\Ss}{\mathscr{S}}
\newcommand{\Af}{\mathfrak{A}}
\newcommand{\Aa}{\mathscr{A}}
\newcommand{\PP}{\mathcal{P}}
\newcommand{\red}{\mathrm{red}}
\newcommand{\Sh}{\mathfrak{Sh}}
\newcommand{\Psh}{\mathfrak{Psh}}
\newcommand{\LRS}{\mathsf{LRS}}
\newcommand{\Sch}{\mathfrak{Sch}}
\newcommand{\Var}{\mathfrak{Var}}
\newcommand{\Rings}{\mathfrak{Rings}}
\DeclareMathOperator{\In}{in}
\DeclareMathOperator{\Ext}{Ext}
\DeclareMathOperator{\Spe}{Sp\acute{e}}
\DeclareMathOperator{\HHom}{\mathscr{H}\!\mathit{om}}
\newcommand{\isoto}{\overset{\sim}{\to}}
\newcommand{\isolongto}{\overset{\sim}{\longrightarrow}}
\newcommand{\Mod}{\mathsf{mod}\mathchar`-}
\newcommand{\MOD}{\mathsf{Mod}\mathchar`-}
\newcommand{\gr}{\mathsf{gr}\mathchar`-}
\newcommand{\qgr}{\mathsf{qgr}\mathchar`-}
\newcommand{\uqgr}{\underline{\mathsf{qgr}}\mathchar`-}
\newcommand{\qcoh}{\mathsf{qcoh}\mathchar`-}
\newcommand{\Alg}{\mathsf{Alg}\mathchar`-}
\newcommand{\coh}{\mathsf{coh}\mathchar`-}
\newcommand{\vect}{\mathsf{vect}\mathchar`-}
\newcommand{\imm}[1][imm]{\hspace{0.75ex}\raisebox{0.58ex}{%
\begin{tikzpicture}[commutative diagrams/every diagram]
\draw[commutative diagrams/.cd, every arrow, every label,hook,{#1}] (0,0ex) -- (2.25ex,0ex);
\end{tikzpicture}}\hspace{0.75ex}}
\newcommand{\dashto}[2]{\smash{\hspace{-0.7em}\begin{tikzcd}[column sep=small,ampersand replacement=\&] {#1} \rar[dashed] \& {#2} \end{tikzcd}\hspace{-0.7em}}}

\usepackage{todonotes}
%\usepackage[notref,notcite]{showkeys}

\title{Hartshorne Ch.~II,\\\S4 Separated and Proper Morphisms}
\author{Takumi Murayama}

\begin{document}
\maketitle
\setcounter{section}{4}
\begin{problem}
  Show that a finite morphism is proper.
\end{problem}
\begin{lemma}\label{lem:finitebasechange}
  Finite morphisms are stable under base extension.
\end{lemma}
\begin{proof}[Proof of Lemma]
  We want to show that if $f\colon X \to S$ is finite, and $g\colon S' \to S$ is any morphism, then $f'\colon X \times_S S' \to S'$ is also finite.
  \par Cover $S$ with open affines $V_i = \Spec B_i$; then $g^{-1}(V_i)$ is a
  cover for $S'$. Now cover these $g^{-1}(V_i)$ with open affines $V_{ij}' =
  \Spec B_{ij}'$; then, $f^{\prime-1}(V_{ij}') = X \times_S V_{ij}' \cong f^{-1}(V_i)
  \times_{V_i} V_{ij}'$ by Thm.~3.3 Step 7. But for each $i$, $f^{-1}(V_i) =
  \Spec A_i \subset X$ for $A_{i}$ that are $B_i$-algebras, finitely
  generated as modules over $B_i$. But since the three
  schemes in this fibre product are affine, we see $f^{-1}(V_i) \times_{V_i}
  V_{ij}' = \Spec (A_i \otimes_{B_i} B_{ij}')$. Since $A_i$ are $B_i$-algebras
  finitely generated as modules over $B_i$, we have a surjection $B_i^{\oplus n} \to
  A_{i}$, and tensoring with $B_{ij}'$ over $B_i$ gives a surjection
  $B_{ij}^{\prime\oplus n} \to A_{i} \otimes_{B_i} B_{ij}'$ by the right-exactness of
  the tensor product, and so $A_{i} \otimes_{B_i} B_{ij}'$ is a
  $B_{ij}'$-algebra, finitely generated as a module over $B_{ij}'$.
\end{proof}
\begin{proof}[Main Proof]
  Let $f\colon X \to Y$ be a finite morphism. A finite morphism is automatically
  of finite type, and is universally closed since finite morphisms are stable
  under base extension by Lemma \ref{lem:finitebasechange}, and finite morphisms
  are closed by Exercise $3.5(b)$.
  \par Thus, it suffices to show $f$ is separated. Let $U_i = \Spec A_i \coloneqq
  f^{-1}(V_i)$ where $V_i = \Spec B_i$ is an affine open cover of $Y$, and $A_i$
  is a $B_i$-algebra finitely generated as a module over $B_i$. Consider the diagonal
  morphism $\Delta \colon X \to X \times_Y X$. By Thm.\ 3.3, $X \times_Y X$ is
  covered by open affines $U_i \times_{V_i} U_i$. Now $U_i = \Delta^{-1}(U_i
  \times_{V_i} U_i)$, and the property of being a closed immersion is local on
  target by our proof of Exercise $3.11(a)$, hence $\Delta$ is a closed
  immersion by Prop.\ $4.1$.
\end{proof}

\begin{problem}
  Let $S$ be a scheme, let $X$ be a reduced scheme over $S$, and let $Y$ be a
  separated scheme over $S$. Let $f$ and $g$ be two $S$-morphisms of $X$ to $Y$
  which agree on an open dense subset of $X$. Show that $f = g$. Give examples
  to show that this result fails if either $(a)$ $X$ is nonreduced, or $(b)$ $Y$
  is nonseparated.
\end{problem}
\begin{proof}
  Consider the map $h \colon X \to Y \times_S Y$ obtained by the universal
  property of fibre products from $f$ and $g$, and let $U$ be the open set in
  $X$ on which $f$ and $g$ agree. Since $X$ is irreducible by Prop.\ $3.1$, $U$
  contains the generic point of $X$.
\end{proof}

\printbibliography
\end{document}
