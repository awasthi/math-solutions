\documentclass[12pt]{article}
\usepackage{geometry}
\geometry{letterpaper}
\usepackage[utf8]{inputenc}
\usepackage[unicode]{hyperref}
\usepackage{amsmath,amsthm,amssymb}
\usepackage{mathtools}
\usepackage{ifpdf}
  \ifpdf
    \setlength{\pdfpagewidth}{8.5in}
    \setlength{\pdfpageheight}{11in}
  \else
\fi

\usepackage{tikz}
\usepackage{tikz-cd}

\usepackage{bm}

\newtheorem{probaux}[subsubsection]{Exercise}
\newtheorem*{claim}{Claim}
\newtheorem*{lemma}{Lemma}%[subsubsection]
\theoremstyle{remark}
\newtheorem*{remark}{Remark}

\renewcommand{\thesubsection}{\arabic{subsection}}
%\renewcommand{\thelemma}{\thesubsubsection\alph{lemma}}

\usepackage{xparse}
\NewDocumentEnvironment{problem}{o}
 {\IfNoValueTF{#1}
   {\probaux\addcontentsline{toc}{subsubsection}{\protect Exercise \thesubsubsection}}
   {\probaux[#1]\addcontentsline{toc}{subsubsection}{\protect Exercise \thesubsubsection}}%
   \ignorespaces}
 {\endprobaux}

\usepackage{shorttoc}
\usepackage[toc]{multitoc}
\renewcommand*\contentsname{List of Solved Exercises}
\usepackage{tocloft}

\newcounter{enumacounter}
\newenvironment{enuma}
{\begin{list}{$(\alph{enumacounter})$}{\usecounter{enumacounter} \parsep=0em \itemsep=0em \leftmargin=2.75em \labelwidth=1.5em \topsep=0em}}
{\end{list}}
\newcounter{enumicounter}
\newenvironment{enumi}
{\begin{list}{$(\roman{enumicounter})$}{\usecounter{enumicounter} \parsep=0em \itemsep=0em \leftmargin=2.25em \labelwidth=2em \topsep=0em}}
{\end{list}}

\DeclareMathOperator{\Aut}{Aut}
\let\Im\relax
\DeclareMathOperator{\Im}{im}
\DeclareMathOperator{\lcm}{lcm}
\DeclareMathOperator{\id}{id}
\newcommand{\GL}{\mathit{GL}}
\newcommand{\PGL}{\mathit{PGL}}
\newcommand{\SL}{\mathit{SL}}
\newcommand{\bracket}[1]{[#1]}
\let\amsamp=&

\title{Selected Solutions to Munkres's Topology, 2nd Ed.}
\author{Takumi Murayama}

\begin{document}
\maketitle
These solutions are the result of taking MAT365 Topology in the Fall of 2012 at Princeton University. This is not a \emph{complete} set of solutions; see the \hyperlink{det.1}{List of Solved Exercises} at the end. Please e-mail \href{mailto:takumim@umich.edu}{\nolinkurl{takumim@umich.edu}} with any corrections.
\pdfbookmark[1]{Contents}{toc}
\begingroup
\setlength{\cftsubsecnumwidth}{2.75em}
\shorttoc{Contents}{2}
\endgroup
\newpage
\part{General Topology}
\setcounter{section}{1}
\section{Topological Spaces and Continuous Functions}
\setcounter{subsection}{12}
\subsection{Basis for a Topology}
\setcounter{subsubsection}{2}
\begin{problem}
  Show that the collection $\mathcal{T}_c$ given in Example $4$ of $\S12$ is a topology on the set $X$. Is the collection
  \begin{equation*}
    \mathcal{T}_\infty = \{U \mid X \setminus U~\text{is infinite or empty or all of}~X\}
  \end{equation*}
  a topology on $X$?
\end{problem}
\begin{proof}
  Recall Example 12.4: Let $X$ be a set; let $\mathcal{T}_c$ be the
  collection of all subsets $U$ of $X$ such that $X \setminus U$ either is
  countable or is all of $X$. We claim this forms a topology on $X$; we will
  follow the numbering for the definition of a topology on p.~76. (1) $X
  \setminus \emptyset = X$ and $X \setminus X = \emptyset$ is countable; (2)
  $X \setminus \bigcup_{\alpha \in A} U_\alpha = \bigcap_{\alpha \in A} (X
  \setminus U_\alpha)$ is countable since it is an intersection of countable sets,
  unless every $U_\alpha = \emptyset$, in which case $X \setminus \bigcup_{\alpha \in
  A} U_\alpha = X$; (3) $X \setminus \bigcap_{\alpha \in A~\text{finite}}
  U_\alpha = \bigcup_{\alpha \in A~\text{finite}} (X \setminus U_\alpha)$ is
  countable since it is the finite union of countable sets, unless every
  $U_\alpha = \emptyset$, in which case $X \setminus
  \bigcap_{\alpha \in A~\text{finite}} U_\alpha = X$.
  \par Now consider $\mathcal{T}_\infty$. It is not a topology, for if we let $X = [-1,1] \subseteq \mathbb{R}$, and let $U_1 = [-1,0)$ and $U_2 = (0,1]$, we see that both $U_1,U_2 \in \mathcal{T}_\infty$, but $X \setminus (U_1 \cup U_2) = \{0\}$, which is not infinite, and so $U_1 \cup U_2 \notin \mathcal{T}_\infty$.
\end{proof}

\setcounter{subsubsection}{4}
\begin{problem}
  Show that if $\mathcal{A}$ is a basis for a topology on $X$, then the topology generated by $\mathcal{A}$ equals the intersection of all topologies on $X$ that contain $\mathcal{A}$. Prove the same if $\mathcal{A}$ is a subbasis.
\end{problem}
\begin{proof}
  Let $\mathcal{T}_\mathcal{A}$ be the topology generated by $\mathcal{A}$, and $\mathcal{T}_I$ be the intersection of all topologies that contain $\mathcal{A}$.
  \par $\mathcal{T}_I \subseteq \mathcal{T}_\mathcal{A}$. This follows from the fact that $\mathcal{T}_\mathcal{A} \supseteq \mathcal{A}$, and so is one of the topologies that is intersected over in the construction of $\mathcal{T}_I$.
  \par $\mathcal{T}_\mathcal{A} \subseteq \mathcal{T}_I$. Let $U \in \mathcal{T}_\mathcal{A}$; by Lemma $13.1$, $U = \bigcup_\alpha A_\alpha$ for some collection $\{A_\alpha\}_\alpha \subseteq \mathcal{A}$. But $U = \bigcup_\alpha A_\alpha \in \mathcal{T}_I$ since each $A_\alpha \in \mathcal{T}_I$.
  \par Now let $\mathcal{A}$ be a subbasis. The proof that $\mathcal{T}_I \subseteq \mathcal{T}_\mathcal{A}$ is identical; it remains to show $\mathcal{T}_\mathcal{A} \subseteq \mathcal{T}_I$. Let $U \in \mathcal{T}_\mathcal{A}$; by definition of the topology generated by $\mathcal{A}$, $U$ is the union of a finite intersection of elements $\{A_\alpha\}_\alpha \subseteq \mathcal{A}$. But then $U \in \mathcal{T}_I$ since each $A_\alpha \in \mathcal{T}_I$.
\end{proof}

\begin{problem}
  Show that the topologies of $\mathbb{R}_\ell$ and $\mathbb{R}_K$ are not comparable.
\end{problem}
\begin{proof}
  $\mathbb{R}_\ell \not\subseteq \mathbb{R}_K$. For $[a,b) \in \mathbb{R}_\ell$, there is no basis element $U \in \mathbb{R}_K$ such that $a \in U, U \subseteq [a,b)$. 
  \par $\mathbb{R}_K \not\subseteq \mathbb{R}_\ell$. For $(-1,1) \setminus K \in
  \mathbb{R}_K$ which contains $0$, there is no basis element $[a,b) \in
  \mathbb{R}_\ell$ such that $0 \in [a,b), [a,b) \subseteq (-1,1) \setminus K$
  by the Archimedean property, that is, for all $\epsilon > 0$, there exists $N
  \in \mathbb{N}$ such that $1/N < \epsilon$.
\end{proof}

\begin{problem}\label{exc:13.7}
  Consider the following topologies on $\mathbb{R}$:
  \begin{align*}
    \mathcal{T}_1 &= \text{the standard topology,}\\
    \mathcal{T}_2 &= \text{the topology of $\mathbb{R}_K$,}\\
    \mathcal{T}_3 &= \text{the finite complement topology,}\\
    \mathcal{T}_4 &= \text{the upper limit topology, having all sets $(a,b]$ as basis,}\\
    \mathcal{T}_5 &= \text{the topology having all sets $(-\infty,a) = \{x \mid x < a\}$ as basis.}
  \end{align*}
  Determine, for each of these topologies, which of the others it contains.
\end{problem}
\begin{proof}
  We claim we have the following Hasse diagram:
  \begin{center}
    \begin{tikzpicture}
      \node (A) at (0,0) {$\mathcal{T}_1$};
      \node (B) at (-1,-1) {$\mathcal{T}_3$};
      \node (C) at (1,-1) {$\mathcal{T}_5$};
      \node (D) at (0,1) {$\mathcal{T}_2$};
      \node (E) at (0,2) {$\mathcal{T}_4$};
      \path[-] (A) edge node{} (B)
      (A) edge node{} (C)
      (A) edge node{} (D)
      (D) edge node{} (E);
    \end{tikzpicture} 
  \end{center}
  \par $\mathcal{T}_3 \subsetneq \mathcal{T}_1$. Inclusion is true since $U \in \mathcal{T}_3 \implies U^c$ finite, and so if we let $U^c = \{x_i\}_{i=1}^n$ with $x_i$ in increasing order, $U = \bigcup_{i=0}^n (x_i,x_{i+1})$, where $x_0 = -\infty,x_{n+1} = \infty$. Inequality follows since for $(a,b)$ such that $-\infty < a,b < \infty$, $\mathbb{R} \setminus (a,b)$ is not finite.
  \par $\mathcal{T}_5 \subsetneq \mathcal{T}_1$. Inclusion is clear since
  $(-\infty,a)$ is of the form $(b,c)$. Inequality follows since for $(b,c) \in
  \mathcal{T}_1$ and $x \in (b,c)$, there is no basis element $(-\infty,a) \in
  \mathcal{T}_5$ such that $x \in (-\infty,a),(-\infty,a) \subseteq (b,c)$ (if
  $b > -\infty$).
  \par $\mathcal{T}_3$ and $\mathcal{T}_5$ are not comparable. $\mathcal{T}_3 \not\subseteq \mathcal{T}_5$ since $\mathbb{R} \setminus \{0\} \in \mathcal{T}_3$, but if we take $x > 0$, which is in this set, there is no basis element $(-\infty,a) \in \mathcal{T}_5$ that contains $x$ but is contained in $\mathbb{R} \setminus \{0\}$. $\mathcal{T}_5 \not\subseteq \mathcal{T}_3$ since $(-\infty,0)^c$ is not finite.
  \par $\mathcal{T}_1 \subsetneq \mathcal{T}_2$ by Lemma $13.4$.
  \par $\mathcal{T}_2 \subsetneq \mathcal{T}_4$. For $(a,b) \in \mathcal{T}_2$ and $x \in (a,b)$, $(a,x] \in \mathcal{T}_4$ and $(a,x] \subseteq (a,b)$. For $(a,b) \setminus K \in \mathcal{T}_2$ and $x \in (a,b) \setminus K$, we note that $x \in (1/(n+1),c]$ where $x < c < 1/n$, $x \in (a,0]$, or $x \in (1,d]$, where $x < d < b$; in all three cases, these sets are subsets of $(a,b) \setminus K$ and are members of $\mathcal{T}_4$. Inequality follows since for $(a,b] \in \mathcal{T}_4$, there is no basis element $U \in \mathcal{T}_2$ such that $b \in U,U \subset (a,b]$.
\end{proof}

\setcounter{subsection}{15}
\subsection{The Subspace Topology}
\setcounter{subsubsection}{7}
\begin{problem}
  If $L$ is a straight line in the plane, describe the topology $L$ inherits as a subspace of $\mathbb{R}_\ell \times \mathbb{R}$ and as a subspace of $\mathbb{R}_\ell \times \mathbb{R}_\ell$. In each case it is a familiar topology.
\end{problem}
\begin{proof}[Solution]
  Note that the basis for $\mathbb{R}_\ell \times \mathbb{R}$ consists of
  elements of the form $[a,b) \times (c,d)$. If $L = \{(x,y) \mid x = x_0\}$,
  then $L \cap [a,b) \times (c,d) = \emptyset$ or $\{x_0\} \times (c,d)$, and
  so defining the map $\varphi\colon L \cap (\mathbb{R}_\ell \times \mathbb{R}) \to \mathbb{R}, \{x_0\} \times (c,d) \mapsto (c,d)$, it is bijective, open, and continuous, and so the topology $L$ inherits is homeomorphic to $\mathbb{R}$ with the standard topology. If $L$ has finite slope, we first note that $L \cap (\mathbb{R}_\ell \times \mathbb{R}) = \{(x,mx+b) \in \mathbb{R}^2 \mid x \in \mathbb{R}\}$, and that the basis for our topology are the sets of the form $\emptyset, [ (a,ma+b),(c,mc+b) ), ( (a,ma+b),(c,mc+b) )$ for $a,c \in \mathbb{R}$ and $a < c$, by Lemma $16.1$. We then define
  \begin{equation*}
    \varphi\colon L \cap (\mathbb{R}_\ell \times \mathbb{R}) \to
    \mathbb{R}_\ell, \quad (a,ma+b) \mapsto a.
  \end{equation*}
  This implies
  \begin{align*}
    ( (a,ma+b),(c,mc+b) ) &\mapsto (a,c),\\
    [ (a,ma+b),(c,mc+b) ) &\mapsto [a,c).
  \end{align*}
  We claim this defines a homeomorphism with $\mathbb{R}_\ell$. Clearly, it is continuous, for the basis elements of $\mathbb{R}_\ell$ have preimages that are basis elements in the topology on $L$. Likewise, it is open since the basis elements of $L$ map to sets that are open in $\mathbb{R}_\ell$ by Lemma $13.4$. Finally this is a bijection since there exists an inverse just by reversing the arrows above.
  \par For $\mathbb{R}_\ell \times \mathbb{R}_\ell$, by following the same steps
  as above if $L = \{(x,y) \mid x = x_0\}$, then $L \cap (\mathbb{R}_\ell \times
  \mathbb{R}_\ell)$ is homeomorphic to $\mathbb{R}_\ell$. For $L$ with $|m| <
  \infty$, we must split it up into two cases. When $m \ge 0$, we have a similar situation as above, except we only have to consider basis elements of the form $[a,b)$; thus, $L \cap (\mathbb{R}_\ell \times \mathbb{R}_\ell)$ is homeomorphic to $\mathbb{R}_\ell$. When $m < 0$, since for every point $(x,y) \in L$, we can find a basis element $[x,a)\times[y,b) \in (\mathbb{R}_\ell \times \mathbb{R}_\ell)$ such that $L \cap [x,a)\times[y,b) = \{(x,y)\}$, and these form the open sets of our new topology by Lemma $16.1$. We see then that the topology on $L$ is homeomorphic to the discrete topology on $\mathbb{R}$.
\end{proof}

\begin{problem}
  Show that the dictionary order topology on the set $\mathbb{R} \times \mathbb{R}$ is the same as the product topology $\mathbb{R}_d \times \mathbb{R}$, where $\mathbb{R}_d$ denotes $\mathbb{R}$ in the discrete topology. Compare this topology with the standard topology on $\mathbb{R}^2$.
\end{problem}
\begin{proof}
  We see that the basis elements of $(\mathbb{R} \times
  \mathbb{R})_\mathrm{lex}$ consist of intervals of the form $(a \times b, c
  \times d)$ for $a < c$, and for $a = c$ and $b < d$, as in Example 14.2.
  These basis elements are open in $\mathbb{R}_d \times \mathbb{R}$ since
  \begin{equation*}
    (a \times b,c\times d) = (a,c) \times \mathbb{R} \cup \{a\} \times (c,\infty)
    \cup \{b\} \times (-\infty,d) \in \mathcal{T}_{\mathbb{R}_d \times
    \mathbb{R}}.
  \end{equation*}
  For the reverse situation, consider the basis elements for $\mathbb{R}_d
  \times \mathbb{R}$; these consist of all $\{a\} \times (b,c)$ since $\{a \mid
    a \in \mathbb{R}\}$ forms a basis for $\mathbb{R}_d$ by Example 13.3. But then, $\{a\} \times (b,c)$ are open in $\mathbb{R} \times \mathbb{R}$ with the order topology since it is of the form $(a \times b, c \times d)$ for $a = c$.
  \par We now compare this to the standard topology on $\mathbb{R}^2$. Since $(a,b) \times (c,d) \in \mathbb{R}_d \times \mathbb{R}$, we see that $\mathbb{R}^2 \subseteq \mathbb{R}_d \times \mathbb{R}$. Moreover, since $\{a\} \times (b,c) \in (\mathbb{R}_d \times \mathbb{R}) \setminus \mathbb{R}^2$, we see that $\mathbb{R}^2 \subsetneq \mathbb{R}_d \times \mathbb{R}$.
\end{proof}

\subsection{Closed Sets and Limit Points}
\setcounter{subsubsection}{1}
\begin{problem}
  Show that if $A$ is closed in $Y$ and $Y$ is closed in $X$, then $A$ is closed in $X$.
\end{problem}
\begin{proof}
  $A$ is closed in $Y$ iff there exists $B \subseteq X$ closed in $X$ such that $A = Y \cap B$ by Theorem $17.2$. But then, $A$ is the intersection of closed sets, and so is closed.
\end{proof}

\begin{problem}
  Show that if $A$ is closed in $X$ and $B$ is closed in $Y$, then $A \times B$ is closed in $X \times Y$.
\end{problem}
\begin{proof}
  We see that $X \setminus A, Y \setminus B$ are open in $X,Y$ respectively by definition of a closed set. By definition of the product topology, $(X \setminus A) \times Y, X \times (Y \setminus B)$ are open in $X \times Y$. We see that $(X \setminus A) \times Y = (X \times Y) \setminus (A \times Y), X \times (Y \setminus B) = (X \times Y) \setminus (X \times B)$, and so $A \times Y, X \times B$ are closed in $X \times Y$. Finally, $A \times B = (A \times Y) \cap (X \times B)$, and so is the intersection of closed sets, i.e., closed.
\end{proof}

\setcounter{subsubsection}{4}
\begin{problem}
  Let $X$ be an ordered set in the order topology. Show that $\overline{(a,b)} \subset [a,b]$. Under what conditions does equality hold?
\end{problem}
\begin{proof}
  Since $(a,b) \subseteq [a,b]$ closed, and by the definition of closure,
  \begin{equation*}
    \overline{(a,b)} =\ \ \bigcap_{\mathclap{K \supseteq (a,b)~\text{closed}}}\
    \ K \subseteq [a,b].
  \end{equation*}
  $\overline{(a,b)} = [a,b] \iff a,b \in \overline{(a,b)} \iff$ any basis
  elements $A \ni a, B \ni b$ intersect $(a,b)$ by Theorem $17.5(b)$. We claim
  that this is equivalent to the fact that there is no immediate successor
  $\alpha$ of $a$ and no immediate predecessor $\beta$ of $b$. If either are the
  case, say for $a$, then choosing $A$ with upper bound $\alpha$ would not
  intersect $(a,b)$, and so equality doesn't hold since $a \notin \overline{(a,b)}$; in the other direction, if neither are the case, we see that, say for $a$, the upper bound of $A$, $\alpha$ would be such that $(a,\alpha)$ is non-empty, and so $A \cap (a,b) \ne \emptyset$, satisfying the condition for Theorem $17.5(b)$. The same argument applies when considering $b$ and $\beta$, and so our claim holds.
\end{proof}

\setcounter{subsubsection}{12}
\begin{problem}\label{exc:17.13}
  Show that $X$ is Hausdorff if and only if the \emph{\textbf{diagonal}} $\Delta = \{x \times x \mid x \in X\}$ is closed in $X \times X$.
\end{problem}
\begin{proof}
  Suppose $\Delta$ is closed in $X \times X$, i.e., the complement $\Delta^c$ is
  open. This is equivalent to for all $(x,y) \in X \times X$ such that $x \ne
  y$, there exists a basis element $U \times V$ of $X \times X$ for $U,V$ open
  in $X$ such that $(x,y) \in U \times V$ but $(U \times V) \cap \Delta =
  \emptyset$. But then, by definition of $\Delta$, this is equivalent to saying
  for all $x,y \in X$ such that $x \ne y$, there exist open neighborhoods $U \ni x$
  and $V \ni y$ such that $U \cap V = \emptyset$, and so $X$ is Hausdorff.
\end{proof}

\setcounter{subsubsection}{15}
\begin{problem}
  Consider the five topologies on $\mathbb{R}$ given in \href{exc:13.7}{Exercise
  $7$ of $\S13$}.
  \begin{enuma}
    \item Determine the closure of the set $K = \{1/n \mid n \in \mathbb{Z}_+\}$ under each of these topologies.
    \item Which of these topologies satisfy the Hausdorff axiom? the $T_1$ axiom?
  \end{enuma}
\end{problem}
\begin{proof}[Solution for $(a)$]
  \par For $\mathcal{T}_3$, $A$ closed $\iff A$ finite or all of $\mathbb{R}$. Since no finite set contains all of $K$, we see that $\mathbb{R}$ is the only closed set containing $K$, and so $\overline{K} = \mathbb{R}$.
  \par For $\mathcal{T}_5$, we claim $\overline{K} = [0,\infty)$. For $x \in [0,\infty)$, the basis elements that contain $x$ are of the form $(-\infty,a)$ for $a > x$. Since $(-\infty,a) \cap K \ne \emptyset$ by the Archimedean property, that is, $\forall \epsilon > 0 \exists N \in \mathbb{N}$ such that $1/N < \epsilon$, $\overline{K} = [0,\infty)$ by Theorem $17.5$.
  \par For $\mathcal{T}_1$, $K' = \{0\}$ by Example 17.8, and so $\overline{K} = K \cup \{0\}$.
  \par For $\mathcal{T}_2$, $K$ is closed since $\mathbb{R} \setminus K = (-\infty,\infty) \setminus K$ is a basis element, and so $\overline{K} = K$.
  \par For $\mathcal{T}_4$, $\overline{K} = K$ since $\mathcal{T}_4$ is finer than $\mathcal{T}_2$, and so $\mathbb{R} \setminus K$ is still open.
\end{proof}
\begin{proof}[Solution for $(b)$]
  $\mathcal{T}_3$ is $T_1$ since any finite point set is closed by definition of the finite complement topology. It is not Hausdorff, for if we choose $U \ni x, V \ni y$ both open, $(U \cap V)^c = U^c \cup V^c$ is finite, where the equality follows from De Morgan's Laws, and so $U \cap V$ is infinite.
  \par $\mathcal{T}_5$ is not Hausdorff and not even $T_1$, for $\mathbb{R} \setminus \{x_0\}$ is not a union of basis elements, and so $\{x_0\}$ is not closed.
  \par $\mathcal{T}_1$ is Hausdorff, for if we have $x,y \in \mathbb{R}$ and $0 < \epsilon < |x-y|/2$, then $(x-\epsilon,x+\epsilon) \cap (y-\epsilon,y+\epsilon) = \emptyset$. Since Hausdorff $\implies T_1$, we see that $\mathcal{T}_1$ is also $T_1$.
  \par Since $\mathcal{T}_2,\mathcal{T}_4$ are both finer than $\mathcal{T}_1$, we see that the open sets constructed above are still open and separate $x,y$, and so $\mathcal{T}_2,\mathcal{T}_4$ are still both Hausdorff and thus $T_1$.
\end{proof}

\subsection{Continuous Functions}
\begin{problem}
  Prove that for functions $f\colon \mathbb{R} \to \mathbb{R}$, the $\epsilon$-$\delta$ definition of continuity implies the open set definition.
\end{problem}
\begin{remark}
  Recall that $f$ is continuous if for every $\epsilon > 0$ and $x_0 \in \mathbb{R}$, there exists a $\delta > 0$ such that $|f(x) - f(x_0)| < \epsilon$ for all $x \in \mathbb{R}$ such that $|x-x_0| < \delta$.
\end{remark}
\begin{proof}
  Consider $x_0 \in \mathbb{R}$, and a corresponding neighborhood $V$ of $f(x_0)$; we then have $V \supseteq (f(x_0)-\epsilon,f(x_0)+\epsilon)$ for some $\epsilon > 0$ since $V$ is open. Then, by hypothesis there exists a $\delta > 0$ such that $f(x) \in (f(x_0)-\epsilon,f(x_0)+\epsilon)$ for all $x \in \mathbb{R}$ such that $x \in (x_0-\delta,x_0+\delta) = U$, which is open. Thus, $f(U) \subseteq V$, and so $f$ is continuous by Theorem $18.1$.
\end{proof}

\setcounter{subsubsection}{11}
\begin{problem}
  Let $F\colon \mathbb{R} \times \mathbb{R} \to \mathbb{R}$ be defined by the equation
  \begin{equation*}
    F(x \times y) = \begin{cases}
      xy/(x^2+y^2) & \text{if}~x \times y \ne 0 \times 0.\\
      0 & \text{if}~x \times y = 0 \times 0.
    \end{cases}
  \end{equation*}
  \begin{enuma}
    \item Show that $F$ is continuous in each variable separately.
    \item Compute the function $g\colon \mathbb{R} \to \mathbb{R}$ defined by $g(x) = F(x \times x)$.
    \item Show that $F$ is not continuous.
  \end{enuma}
\end{problem}
\begin{proof}[Proof of $(a)$]
  Since $F$ is symmetric in interchanging $x \leftrightarrow y$, we only have to
  prove $\forall y_0 \in Y$, $h(x) = F(x \times y_0)$ is continuous as a
  function $\mathbb{R} \to \mathbb{R}$. For $y_0 = 0$, this is trivially true
  for the image of $h$ is $(0,0)$ with preimage $\mathbb{R}$. Now suppose $y_0
  \ne 0$; then we have $h(x) = xy_0/(x^2+y_0^2)$. This is continuous since
  $xy_0$ and $x^2 + y_0^2$ are both continuous, and so their quotient is also
  continuous (since also $x^2 + y_0^2 \ne 0$), using the $\epsilon$-$\delta$
  definition of continuity (see Theorem 21.5).
\end{proof}
\begin{proof}[Proof of $(b)$]
  Since $F(x \times x)$ for $x \ne 0$ equals $x^2/(x^2+x^2) = x^2/2x^2 = 1/2$, we have
  \begin{equation*}
    g(x) = \begin{cases}
      1/2 & \text{if}~x \ne 0.\\
      0 & \text{if}~x = 0.
    \end{cases}\qedhere
  \end{equation*}
\end{proof}
\begin{proof}[Proof of $(c)$]
  We claim $F(x \times y)$ is not continuous along the line $L = \{x=y\}$ at $(0,0)$,
  i.e., $F\rvert_L$ is not continuous at $(0,0)$; this
  suffices by Theorem $18.2(d)$. Note that the line $L$ in the subspace
  topology is homeomorphic to $\mathbb{R}$, where the homeomorphism is given by
  either of the coordinate projection maps. Now the preimage of the closed set
  $\{1/2\} \subseteq \mathbb{R}$ is $L \setminus \{(0,0)\}$, which is not
  closed since $\mathbb{R} \setminus \{0\}$ is not closed, hence $F\rvert_L$ is
  not continuous, and neither is $F$.
\end{proof}

\subsection{The Product Topology}
\setcounter{subsubsection}{5}
\begin{problem}
  Let $\mathbf{x}_1,\mathbf{x}_2,\ldots$ be a sequence of the points of the
  product space $\prod X_\alpha$. Show that this sequence converges to the point
  $\mathbf{x}$ if and only if the sequence $\pi_\alpha(\mathbf{x}_1),$
  $\pi_\alpha(\mathbf{x}_2),$ $\ldots$ converges to $\pi_\alpha(\mathbf{x})$ for each $\alpha$. Is this fact true if one uses the box topology instead of the product topology?
\end{problem}
\begin{proof}
  Suppose $\{\mathbf{x}_i\} \to \mathbf{x}$, and fix some index $\gamma$. Then,
  for any neighborhood $U_\gamma \ni \pi_\gamma(\mathbf{x})$, letting
  $U = \prod U_\alpha$ where $U_\alpha = X_\alpha$ for all $\alpha \ne \gamma$,
  there exists $N \in \mathbb{N}$ such that $\mathbf{x}_i \in U$ for all $i \ge
  N$, and so $\pi_\gamma(\mathbf{x}_i) \in \pi_\gamma(U) = U_\gamma$ for all $i
  \ge N$, i.e., $\{\pi_\gamma(\mathbf{x}_i)\} \to \pi_\gamma(\mathbf{x})$.
  Note that this direction does not depend on the topology being the product or box topology.
  \par In the other direction, suppose $\{\pi_\alpha(\mathbf{x}_i)\} \to \pi_\alpha(\mathbf{x})$ for all $\alpha$. We take an arbitrary neighborhood $V$ of $\mathbf{x} \in \prod X_\alpha$; it then contains a basis element of $\prod X_\alpha$ containing $\mathbf{x}$, which is a product of open sets $\prod U_\alpha$. In the case of the product topology, there then exist only finite $U_\alpha \subsetneq X_\alpha$, and for these open sets there exist $N_\alpha \in \mathbb{N}$ such that $\pi_\alpha(\mathbf{x}_i) \in U_\alpha$ for all $i \ge N_\alpha$ for each $\alpha$. $N_\alpha = 1$ works for all other $\alpha$. Thus, we can take $N = \max(N_\alpha)$; then, $\mathbf{x}_i \in \prod U_\alpha \subseteq V$ for all $i \ge N$.
  \par We construct a counterexample for this direction in the case of the box
  topology. Let $\mathbb{R}^\mathbb{N}$ be the box topology on the product of
  copies of $\mathbb{R}$ indexed by $\mathbb{N}$, and let
  \begin{equation*}
    \mathbf{x}_i \coloneqq ( \tfrac{1}{i}, \tfrac{1}{i},\tfrac{1}{i},\ldots).
  \end{equation*}
  Then, for each $\alpha \in \mathbb{N}$, $\{\pi_\alpha(\mathbf{x}_i)\} \to
  (0,0,0,\ldots) \eqqcolon \mathbf{x}$, but this sequence does not converge in
  the box topology, for the open set
  \begin{equation*}
    \prod_{i \in \mathbb{N}} ( -\tfrac{1}{i},\tfrac{1}{i} ) = (-1,1)
    \times ( -\tfrac{1}{2},\tfrac{1}{2} ) \times (
    -\tfrac{1}{3},\tfrac{1}{3} ) \times \cdots
  \end{equation*}
  in the box topology contains $\mathbf{x} = (0,0,0,\ldots)$, but does not contain
  any $\mathbf{x}_i$.
\end{proof}

\subsection{The Metric Topology}
\setcounter{subsubsection}{3}
\begin{problem}
  Consider the product, uniform, and box topologies on $\mathbb{R}^\omega$.
  \begin{enuma}
  \item In which topologies are the following functions from $\mathbb{R}$ to $\mathbb{R}^\omega$ continuous?
    \begin{align*}
      f(t) &= (t,2t,3t,\ldots),\\
      g(t) &= (t,t,t,\ldots),\\
      h(t) &= (t, \tfrac{1}{2}t,\tfrac{1}{3}t,\ldots).
    \end{align*}
  \item In which topologies do the following sequences converge?
    \begin{alignat*}{4}
      \mathbf{w}_1 &= (1,1,1,1,\ldots), \qquad& \mathbf{x}_1 &= (1,1,1,1,\ldots),\\
      \mathbf{w}_2 &= (0,2,2,2,\ldots), \qquad& \mathbf{x}_2 &= (0,\tfrac{1}{2},\tfrac{1}{2},\tfrac{1}{2},\ldots),\\
      \mathbf{w}_3 &= (0,0,3,3,\ldots), \qquad& \mathbf{x}_3 &= (0,0,\tfrac{1}{3},\tfrac{1}{3},\ldots),\\
      &\ldots & &\ldots\\
      \mathbf{y}_1 &= (1,0,0,0,\ldots), \qquad& \mathbf{z}_1 &= (1,1,0,0,\ldots),\\
      \mathbf{y}_2 &= (\tfrac{1}{2},\tfrac{1}{2},0,0,\ldots), \qquad& \mathbf{z}_2 &= (\tfrac{1}{2},\tfrac{1}{2},0,0,\ldots),\\
      \mathbf{y}_3 &= (\tfrac{1}{3},\tfrac{1}{3},\tfrac{1}{3},0,\ldots), \qquad& \mathbf{z}_3 &= (\tfrac{1}{3},\tfrac{1}{3},0,0,\ldots),\\
      &\ldots & &\ldots
    \end{alignat*}
  \end{enuma}
\end{problem}
\begin{proof}[Solution for $(a)$]
  For the product topology, by Theorem $19.6$, $f,g,h$ are all continuous since each coordinate function is continuous. This is because if an open set in the image of a coordinate function is $(a,b)$, its preimage would still be in the form $(a',b') \subseteq \mathbb{R}$ where $a',b'$ are determined by the linear equations defining $f,g,h$ above.
  \par Now consider the uniform topology. Note by Theorem $21.1$ we can use the familiar $\epsilon$-$\delta$ definition for continuity since our spaces both are metric spaces. We claim $f$ is not continuous. For, suppose it is continuous. Then, given $\epsilon > 0$ and $x \in \mathbb{R}$, there exists $\delta > 0$ such that $|x-y| < \delta \implies |f(x) - f(y)| = \sup_n[\min(n|x-y|,1)] < \epsilon$. But, this is a contradiction since for $n$ large, $\min(n|x-y|,1) = 1$, and so is always greater than $\epsilon$. Now consider $g$. $g$ is continuous since given $\epsilon > 0$ and $x \in \mathbb{R}$, we let $\delta < \min(\epsilon,1)$ and therefore have $|x-y| < \delta \implies |f(x) - f(y)| = \sup_n[\min(|x-y|,1)] = \min(|x-y|,1) < \min(\epsilon,1) \le \epsilon$. $h$ is also continuous since given $\epsilon > 0$ and $x \in \mathbb{R}$, we let $\delta < \min(\epsilon,1)$ and therefore have $|x-y| < \delta \implies |f(x) - f(y)| = \sup_n[\min(|x-y|/n,1)] \le \min(|x-y|,1) < \min(\epsilon,1) \le \epsilon$.
  \par For the box topology, since the box topology is finer than the uniform topology by Theorem $20.4$, we see that $f$ is not continuous. For, if $V$ open in the uniform topology has preimage that is not open in $\mathbb{R}$, $V$ is still open in the box topology and still has the same non-open preimage. Next, by Example $19.2$, we see that $g$ is not continuous. Last, for $h$, we choose
  \begin{equation*}
    B = (-1,1) \times ( -\tfrac{1}{2^2}, \tfrac{1}{2^2} ) \times
    ( -\tfrac{1}{3^2}, \tfrac{1}{3^2} ) \times \cdots,
  \end{equation*}
  and suppose its preimage $h^{-1}(B)$ is open. This implies $h((-\delta,\delta)) \subseteq B$, and so applying $\pi_n$ gives
  \begin{equation*}
    h_n((-\delta,\delta)) = ( -\tfrac{\delta}{n},\tfrac{\delta}{n} ) \subseteq ( -\tfrac{1}{n^2},\tfrac{1}{n^2} )
  \end{equation*}
  for all $n$, a contradiction.
\end{proof}
\begin{proof}[Solution for $(b)$]
  We note that since the product topology is Hausdorff by Theorem $19.4$ and
  both the uniform and box topologies are finer than the product topology by
  Theorems $19.1$ and $20.4$, if a sequence converges to a point $\mathbf{p}$ in one topology, it must converge to the same point in the finer topologies. For, if the sequence converges to $\mathbf{q}$ in the finer topology, then it also converges to $\mathbf{q}$ in the coarser topology, and by the Hausdorff property $\mathbf{p} = \mathbf{q}$.
  \par Consider $\mathbf{w}_n$. For the product topology, we recall that any
  basic open set $U = \prod U_\alpha \ni 0$ is the product of finitely many open subsets of $\mathbb{R}$ with infinitely many copies of $\mathbb{R}$. Letting $N$ be the largest $\alpha$ such that $U_\alpha \subsetneq \mathbb{R}$, we see that $\mathbf{w}_n \in U$ for all $n > N$ since the first $N$ components are zero, and the rest are trivially in the remaining copies of $\mathbb{R}$ of $U$. Thus, $\mathbf{w}_n \to 0$ in the product topology. Now we only have to check if the sequence converges to zero in the other topologies by the above. In the uniform topology, $\overline{\rho}(\mathbf{w}_n,0) = 1$ for all $n$, and so the sequence does not converge. For, if we choose any ball $U = B(0,r) \subseteq \mathbb{R}^\omega$ for $r < 1$, $\mathbf{w}_n \notin U$ for all $n$. Finally, since the box topology is finer than the uniform topology by Theorem $20.4$, we see that this same open set $U$ is such that $\mathbf{w}_n \notin U$ for all $n$, and so $\mathbf{w}_n$ does not converge in the box topology, either.
  \par Consider $\mathbf{x}_n$ and $\mathbf{y}_n$. We claim they both converge to zero in the uniform topology. For any open set $0 \in U \subseteq \mathbb{R}^\omega$ in the uniform topology, we can find $B(0,\epsilon)$ such that $B(0,\epsilon) \subseteq U$; then, we can find $N$ such that $1/N < \epsilon$. We then see that $\mathbf{x}_n,\mathbf{y}_n \in B(0,\epsilon) \subseteq U$ for all $n \ge N$, and so $\mathbf{x}_n,\mathbf{y}_n \to 0$ in the uniform topology. Moreover, since the uniform topology is finer than the product topology, we see that this implies $\mathbf{x}_n,\mathbf{y}_n \to 0$ in the product topology as well. For the box topology, though, we see that neither sequence converges. For, we can construct the set $0 \in U = \prod_{n=1}^\infty (-1/n,1/n)$ (where we only consider sets containing zero by the above), which does not contain $\mathbf{x}_n,\mathbf{y}_n$ for any $n$.
  \par For $\mathbf{z}_n$, we see that for any open set $0 \in U = \prod U_\alpha \subseteq \mathbb{R}^\omega$ in the box topology, for $N$ large enough $1/n \in U_1,U_2$ for $n \ge N$, and so $\mathbf{z}_n \in U$ for all $n \ge N$, since by hypothesis $0 \in U$, the third component onwards of $\mathbf{z}_n$ are always in their respective $U_\alpha$. Thus, $\mathbf{z}_n \to 0$ in the box topology; since the box topology is finer than both the uniform and product topologies, we see that this implies $\mathbf{z}_n$ converges in the other two topologies as well.
\end{proof}

\begin{problem}
  Let $\mathbb{R}^\infty$ be the subset of $\mathbb{R}^\omega$ consisting of all sequences that are eventually zero. What is the closure of $\mathbb{R}^\infty$ in $\mathbb{R}^\omega$ in the uniform topology? Justify your answer.
\end{problem}
\begin{proof}[Solution]
  We claim that $A = \overline{\mathbb{R}^\infty}$ is the set of all sequences that converge to zero; we denote this latter set by $X$. It suffices to show by Theorem $17.5$ that $x \in X$ if and only if every basis element $U \ni x$ intersects $A$. First suppose $x \in X$ and let $U \ni x$ be a basis element in the uniform topology; we then see that we can find an open ball $B(x,\epsilon) \subseteq U$. We know we can find $N \in \mathbb{N}$ such that $|x_n| < \epsilon$ for all $n \ge N$ by the definition of convergence. Then, define $y$ such that $y_n = x_n$ for all $n < N$, and zero otherwise; this means $y \in A$. Then, $\overline{\rho}(x,y) < \epsilon$, and so $y \in B(x,\epsilon) \cap A$.
  \par Now suppose $x \notin X$; it suffices to find a basis element containing $x$ that does not intersect $A$. Since $x \notin X$, there exists a ball $B(0,\epsilon) \subseteq \mathbb{R}$ such that $\{x_n\}_{n \ge N} \not\subseteq B(0,\epsilon)$ for any $N$. The ball $B(x,\epsilon/2)$, then, does not intersect $A$, since for any $y \in B(x,\epsilon/2)$, it is not the case that $y_n = 0$ for all $n \ge N$ for some $N$.
\end{proof}

\begin{problem}
  Let $\overline{\rho}$ be the uniform metric on $\mathbb{R}^\omega$. Given $\mathbf{x} = (x_1,x_2,\ldots) \in \mathbb{R}^\omega$ and given $0 < \epsilon < 1$, let
  \begin{equation*}
    U(\mathbf{x},\epsilon) = (x_1-\epsilon,x_1+\epsilon) \times \cdots \times (x_n-\epsilon,x_n+\epsilon) \times \cdots .
  \end{equation*}
  \begin{enuma}
  \item Show that $U(\mathbf{x},\epsilon)$ is not equal to the $\epsilon$-ball $B_{\overline{\rho}}(\mathbf{x},\epsilon)$.
  \item Show that $U(\mathbf{x},\epsilon)$ is not even open in the uniform topology.
  \item Show that
    \begin{equation*}
      B_{\overline{\rho}}(\mathbf{x},\epsilon) = \bigcup_{\delta < \epsilon} U(\mathbf{x},\delta).
    \end{equation*}
  \end{enuma}
\end{problem}
\begin{proof}[Proof of $(a)$]
  Consider the point
  \begin{equation*}
    \mathbf{y} = (y_n)_{n \in \mathbb{N}}, \quad y_n = x_n + \epsilon \sum_{k=1}^n
    \frac{1}{2^k} = x_n + \epsilon\left( 1 - \frac{1}{2^n} \right).
  \end{equation*}
  $y_n \in (x_n-\epsilon,x_n+\epsilon)$ for all $n$ implies $\mathbf{y} \in U(\mathbf{x},\epsilon)$, while $\mathbf{y} \notin B_{\overline{\rho}}(\mathbf{x},\epsilon)$ since
  \begin{equation*}
    \overline{\rho}(\mathbf{x},\mathbf{y}) = \sup_n \overline{d}(x_n,y_n) = \epsilon.\qedhere
  \end{equation*}
\end{proof}
\begin{proof}[Proof of $(b)$]
  $U(\mathbf{x},\epsilon)$ is not open since the point $\mathbf{y}$ in $(a)$ has no neighborhood contained in $B_{\overline{\rho}}(\mathbf{x},\epsilon)$. For, suppose $B_{\overline{\rho}}(\mathbf{y},\delta) \subseteq U(\mathbf{x},\epsilon)$. We can find $N$ such that
  \begin{equation*}
    \frac{\delta}{2} > \epsilon \sum_{k=N+1}^\infty \frac{1}{2^k},
  \end{equation*}
  since $\sum 1/2^k$ converges, and so its tail becomes infinitesimally small. We see that then, defining $\mathbf{y}'$ such that $y'_n = y_n$ for all $n \ne N$ and $y'_N = y_N + \delta/2$, $\mathbf{y}' \in B_{\overline{\rho}}(\mathbf{y},\delta)$ but $\mathbf{y}' \notin U(\mathbf{x},\epsilon)$ since $y'_n = y_n + \delta/2 > y_n + \epsilon \sum_{k=N+1}^\infty \frac{1}{2^k} = x_n + \epsilon$, a contradiction.
\end{proof}
\begin{proof}[Proof of $(c)$]
  The $\supseteq$ direction is clear, since each $U(\mathbf{x},\delta) \subseteq B_{\overline{\rho}}(\mathbf{x},\epsilon)$ by the fact that $\delta < \epsilon$. Now suppose $\mathbf{z} \in B_{\overline{\rho}}(\mathbf{x},\epsilon)$; if $\overline{\rho}(\mathbf{x},\mathbf{z}) = \xi$, then we can find $\delta \in (\xi,\epsilon)$ so that $\mathbf{z} \in U(\mathbf{x},\xi)$, i.e., $B_{\overline{\rho}}(\mathbf{x},\epsilon) \subseteq \bigcup_{\delta < \epsilon} U(\mathbf{x},\delta)$.
\end{proof}

\setcounter{subsubsection}{7}
\begin{problem}
  Let $X$ be the subset of $\mathbb{R}^\omega$ consisting of all sequences $\mathbf{x}$ such that $\sum x_i^2$ converges. Then the formula
  \begin{equation*}
    d(\mathbf{x},\mathbf{y}) = \left[ \sum_{i=1}^\infty (x_i-y_i)^2 \right]^{1/2}
  \end{equation*}
  defines a metric on $X$. On $X$ we have the three topologies it inherits from the box, uniform, and product topologies on $\mathbb{R}^\omega$. We have also the topology given by the metric $d$, which we call the $\ell^2$-\emph{topology}.
  \begin{enuma}
    \item Show that on $X$, we have the inclusions
      \begin{center}
        box topology $\supset$ $\ell^2$-topology $\supset$ uniform topology.
      \end{center}
    \item The set $\mathbb{R}^\infty$ of all sequences that are eventually zero is contained in $X$. Show that the four topologies that $\mathbb{R}^\infty$ inherits as a subspace of $X$ are all distinct.
    \item The set
      \begin{equation*}
        H = \prod_{n \in \mathbb{Z}_+} [0,1/n]
      \end{equation*}
      is contained in $X$; it is called the \emph{Hilbert cube}. Compare the four topologies that $H$ inherits as a subspace of $X$.
  \end{enuma}
\end{problem}
\begin{proof}[Proof of $(a)$]
  box topology $\supset$ $\ell^2$-topology. Let $\mathcal{U} \ni \mathbf{x}$ be a basis element in the $\ell^2$-topology. Then, there exists a basis element $U = B_d(\mathbf{x},\epsilon) \subseteq \mathcal{U}$ of the $\ell^2$-topology. We claim that $V = X \cap \prod (x_i-\epsilon/2^{i/2},x_i+\epsilon/2^{i/2}) \ni \mathbf{x}$ basis element of the box topology is contained in $U$. Suppose $\mathbf{y} \in V$; then
  \begin{equation*}
    d(\mathbf{x},\mathbf{y})^2 = \sum_{i=1}^\infty (x_i-y_i)^2 < \sum_{i=1}^\infty \frac{\epsilon^2}{2^i} = \epsilon^2 \implies \mathbf{y} \in U,
  \end{equation*}
  i.e., $V \subseteq U \subseteq \mathcal{U}$, and box topology $\supset$ $\ell^2$-topology by Lemma $13.3$.
  \par $\ell^2$-topology $\supset$ uniform topology. Let $\mathcal{U} \ni \mathbf{x}$ be a basis element in the uniform topology. Then, there exists a basis element $U = X \cap B_{\overline{\rho}}(\mathbf{x},\epsilon) \subseteq \mathcal{U}$ of the uniform topology. We claim $V = B_d(\mathbf{x},\epsilon)$ basis element of the $\ell^2$-topology is contained in $U$. If $\epsilon > 1$, then trivially $V \subseteq U = X$, so we assume $\epsilon \le 1$. Suppose $\mathbf{y} \in V$; then
  \begin{equation*}
    \overline{\rho}(\mathbf{x},\mathbf{y}) = \sup |x_i-y_i| \le d(\mathbf{x},\mathbf{y}) < \epsilon \implies \mathbf{y} \in U,
  \end{equation*}
  i.e., $V \subseteq U \subseteq \mathcal{U}$, and $\ell^2$-topology $\supset$ uniform topology by Lemma $13.3$.
\end{proof}
\begin{proof}[Proof of $(b)$]
  By $(a)$ and Theorem $20.4$, we have the inclusions
  \begin{center}
    box topology $\supset$ $\ell^2$-topology $\supset$ uniform topology $\supset$ product topology,
  \end{center}
  since if $\mathcal{T}_1 \supset \mathcal{T}_2$ in the ambient space, we would also have $\mathcal{T}_1' \supset \mathcal{T}_2'$, where $\mathcal{T}_i'$ are the subspace topologies induced by $\mathcal{T}_i$ on $X$, by the fact that the open sets in the subspace $X$ are the open sets of the ambient space intersected with $X$. We claim these are strict inclusions.
  \par box topology $\supsetneq$ $\ell^2$-topology. Consider the open set $U = \mathbb{R}^\infty \cap \prod (-1/i,1/i) \ni 0$ in the box topology. Consider any neighborhood $\mathcal{V} \ni 0$; then, there exists some $V = \mathbb{R}^\infty \cap B_d(0,\epsilon)$ for $\epsilon > 0$ contained in $\mathcal{V}$. We can find $N \in \mathbb{N}$ such that $1/N < \epsilon$, and so $\mathbf{x}$ such that $x_i = 0$ for all $i$ except $x_N = 1/N$ is contained in $V$ since $d(0,\mathbf{x}) = 1/N < \epsilon$ and therefore $\mathcal{V}$, but not in $U$. Hence, $U$ is open in the box topology but not in the $\ell^2$-topology by p.~78, and so box topology $\supsetneq$ $\ell^2$-topology.
  \par $\ell^2$-topology $\supsetneq$ uniform topology. Consider the open set $U = \mathbb{R}^\infty \cap B_d(0,1)$ in the $\ell^2$-topology. Consider any neighborhood $\mathcal{V} \ni 0$; then, there exists some $V = \mathbb{R}^\infty \cap B_{\overline{\rho}}(0,\epsilon)$ for $\epsilon > 0$ contained in $\mathcal{V}$. We can find $N \in \mathbb{N}$ such that $N\epsilon^2 > 4$, and so $\mathbf{x}$ such that $x_i = \epsilon/2$ for all $1 \le i \le N$ and $0$ otherwise is contained in $V$ and therefore $\mathcal{V}$, yet
  \begin{equation*}
    d(\mathbf{x},0)^2 = \sum_{i=1}^\infty x_i^2 = N\frac{\epsilon^2}{4} > 1 \implies d(\mathbf{x},0) > 1,
  \end{equation*}
  and so $\mathbf{x} \notin U$. Hence, $U$ is open in the $\ell^2$-topology but not in the uniform topology by p.~78, and so $\ell^2$-topology $\supsetneq$ uniform topology.
  \par uniform topology $\supsetneq$ product topology. Consider the open set $U = \mathbb{R}^\infty \cap B_{\overline{\rho}}(0,1)$ in the uniform topology. Consider any $V = \mathbb{R}^\infty \cap \prod U_\alpha \ni 0$ where $U_\alpha = \mathbb{R}$ for all but finitely many $\alpha$. Let $N$ be such that $U_N = \mathbb{R}$; then, $\mathbf{x}$ such that $x_i = 0$ for all $i$ except $|x_N| \ge 1$ is in $V$ but not in $U$. Hence, $U$ is open in the uniform topology but not in the product topology by p.~78, and so uniform topology $\supsetneq$ product topology.
\end{proof}
\begin{proof}[Solution for $(c)$]
  We claim that
  \begin{center}
    box topology $\supsetneq$ ($\ell^2$-topology $=$ uniform topology $=$ product topology),
  \end{center}
  i.e., the box topology is strictly finer than the other topologies, which are equal.
  \par To show the equality, it suffices to show product topology $\supset$ $\ell^2$-topology, for then we have $\ell^2$-topology $\supset$ uniform topology $\supset$ product topology $\supset$ $\ell^2$-topology by the same argument as in $(b)$, and so we must have equality throughout. So, consider $\mathcal{U} \ni \mathbf{x}$ open in the $\ell^2$-topology; then, we can find a basis element $U = H \cap B_d(\mathbf{x},\epsilon) \subseteq \mathcal{U}$ of the $\ell^2$-topology for some $\epsilon > 0$. Let $\delta = \epsilon/[\zeta(2)+1]^{1/2}$, and choose $N$ such that $\sum_{i=N}^\infty 1/i^2 < \delta^2$, which exists since $|\zeta(2)| < \infty$. We claim that
  \begin{equation*}
    V = H \cap \left[\prod_{i=1}^{N-1} \left(x_i-\frac{\delta}{i},x_i+\frac{\delta}{i}\right) \times \prod_{i=N}^\infty \mathbb{R}\right] \ni \mathbf{x},
  \end{equation*}
  basis element of the product topology is contained in $U$. Suppose $\mathbf{y} \in V$; then
  \begin{equation*}
    d(\mathbf{x},\mathbf{y})^2 = \sum_{i=1}^\infty (x_i-y_i)^2 < \delta^2 \sum_{i=1}^N \frac{1}{i^2} + \sum_{i=N}^\infty \frac{1}{i^2} < \delta^2[\zeta(2)+1] = \epsilon^2 \implies \mathbf{y} \in U,
  \end{equation*}
  i.e., $V \subseteq U \subseteq \mathcal{U}$, and product topology $\supset$ $\ell^2$-topology by Lemma $13.3$. Thus, we have the equality desired.
  \par It remains to show the box topology is strictly finer than the other topologies; since the other topologies are equal it suffices to show box topology $\supsetneq$ $\ell^2$-topology. But the open set $U = H \cap \prod (-1/i,1/i) \ni 0$ is open in the box topology but not the $\ell^2$-topology by the same argument as the proof that box topology $\supsetneq \ell^2$-topology in $(b)$, and so we are done.
\end{proof}

\subsection{The Metric Topology (continued)}
\begin{problem}
  Let $A \subseteq X$. If $d$ is a metric for the topology of $X$, show that $d|_{A \times A}$ is a metric for the subspace topology on $A$.
\end{problem}
\begin{proof}
  Clearly $d|_{A \times A}$ is a metric since it inherits all the properties for a metric from the metric $d$ for $X$; it therefore suffices to show that every basis element for the subspace topology on $A$ contains some open ball defined by $d|_{A \times A}$, and vice versa, by Lemma $13.3$.
  \par So, suppose $B$ is a basis element for the metric topology on $A$; it
  equals $B_{d|_{A \times A}}(x,r)$ for some $x \in A$ and $r \in \mathbb{R}$.
  But then, $B = B_d(x,r) \cap A$ by definition, and so the subspace topology is
  finer than the metric topology.
  \par Conversely, suppose $B$ is a basis element for the subspace topology on
  $A$; it equals $B_d(x,r) \cap A$ for some basis element $B_d(x,r)$ of $X$. Let
  $y \in B_d(x,r) \cap A$; we see that the set $B_{d|_{A \times A}}(y,r-d(x,y))
  \subseteq A \cap B_d(x,r)$ is a basis element for the metric topology
  contained in $A \cap B_d(x,r)$, since $z \in B_{d|_{A \times A}}(y,r-d(x,y))$
  is such that
  \begin{equation*}
    d(z,x) \le d(z,y) + d(x,y) = d|_{A \times A}(z,y) + d(x,y) < r-d(x,y)
    +d(x,y) = r.
  \end{equation*}
  Thus, the metric topology is finer than the subspace topology. Combining the
  two inclusions, we see the topologies are equal.
\end{proof}

\begin{problem}\label{exc:21.2}
  Let $X$ and $Y$ be metric spaces with metrics $d_X$ and $d_Y$, respectively.
  Let $f \colon X \to Y$ have the property that for every pair of points $x_1,x_2$ of $X$,
  \begin{equation*}
    d_Y(f(x_1),f(x_2)) = d_X(x_1,x_2).
  \end{equation*}
  Show that $f$ is an imbedding. It is called an \emph{\textbf{isometric imbedding}} of $X$ in $Y$.
\end{problem}
\begin{proof}
  We first show $f$ is injective:
  \begin{equation*}
    f(x_1) = f(x_2) \implies d_Y(f(x_1),f(x_2)) = 0 \implies d_X(x_1,x_2) = 0 \implies x_1 = x_2
  \end{equation*}
  by properties of metrics, and so we have an injective map.
  \par Now we consider the map $f'\colon X \to \Im(X) \subseteq Y$, which is a bijection; it suffices to show that $f',f'^{-1}$ are continuous to show $f$ is an imbedding. Let $x \in X$ and $\epsilon > 0$ be given; then, letting $\delta = \epsilon$, we have
  \begin{equation*}
    d_X(x,y) < \delta \implies d_Y(f'(x),f'(y)) < \epsilon,
  \end{equation*}
  and so $f$ is continuous. Given $y \in Y$ and $\epsilon > 0$, letting $\delta = \epsilon$ gives
  \begin{equation*}
    d_Y(x,y) = d_Y(f(a),f(b)) < \delta \implies d_X(f'^{-1}(x),f'^{-1}(y)) = d_X(a,b) < \epsilon,
  \end{equation*}
  where $a,b$ exist by the bijectivity of $f$, and so $f'^{-1}$ is continuous.
\end{proof}

\begin{problem}
  Let $X_n$ be a metric space with metric $d_n$, for $n \in \mathbb{Z}_+$.
  \begin{enuma}
    \item Show that
      \begin{equation*}
        \rho(x,y) = \max\{d_1(x_1,y_1),\ldots,d_n(x_n,y_n)\}
      \end{equation*}
      is a metric for the product space $X_1 \times \cdots \times X_n$.
    \item Let $\overline{d}_i = \min\{d_i,1\}$. Show that
      \begin{equation*}
        D(x,y) = \sup\{\overline{d}_i(x_i,y_i)/i\}
      \end{equation*}
      is a metric for the product space $\prod X_i$.
  \end{enuma}
\end{problem}
\begin{proof}[Proof of $(a)$]
  $\rho$ satisfies properties $(1)$ and $(2)$ on p.~119 since the components do; it then suffices to show the triangle inequality. We first have $d_i(x_i,z_i) \le d(x_i,y_i) + d(y_i,z_i)$ for all $i$. Then, by definition of $\rho$, $d_i(x_i,z_i) \le \rho(x,y) + \rho(y,z)$ for all $i$. But since this is true for all $i$, we have that $\rho(x,z) \le \rho(x,y) + \rho(y,z)$.
  \par We now show that this defines a metric for the product space. First let $B = \prod U_i$ be a basis element of $\prod X_i$, and let $\mathbf{x} \in B$. For each $i$, there is an $\epsilon_i$ such that $B_{d_i}(x_i,\epsilon_i) \subseteq U_i$. Choosing $\epsilon = \min\{\epsilon_1,\ldots,\epsilon_n\}$, we see that $B_\rho(\mathbf{x},\epsilon) \subseteq B$, since if $\mathbf{y} \in B_\rho(\mathbf{x},\epsilon)$, $d_i(x_i,y_i) \le \rho(\mathbf{x},\mathbf{y}) < \epsilon \le \epsilon_i$, and so $\mathbf{y} \in \prod U_i$ as desired. Thus the metric topology is finer than the product topology.
  \par Conversely, let $B_\rho(\mathbf{x},\epsilon)$ be a basis element in the metric topology; since it is the product $B_{d_i}(x_i,\epsilon)$, we see that the product topology is finer than the metric topology. These two facts imply the two topologies are equal.
\end{proof}
\begin{proof}[Proof of $(b)$]
  $D$ satisfies properties $(1)$ and $(2)$ on p.~119 since the components do; it then suffices to show the triangle inequality. We first have
  \begin{equation*}
    \frac{\overline{d}_i(x_i,z_i)}{i} \le \frac{\overline{d}_i(x_i,y_i)}{i} + \frac{\overline{d}_i(y_i,z_i)}{i} \le D(x,y) + D(y,z)
  \end{equation*}
  for all $i$. But since this is true for all $i$, we have that
  \begin{equation*}
    D(x,z) = \sup\left\{\frac{\overline{d}_i(x_i,z_i)}{i}\right\} \le D(x,y) + D(y,z).
  \end{equation*}
  \par We now show that this defines a metric for the product space. Let $U$ be open in the metric topology and let $\mathbf{x} \in U$; choose an open ball $B_D(\mathbf{x},\epsilon) \subseteq U$. Choose $N$ such that $1/N < \epsilon$, and let
  \begin{equation*}
    V = B_{\overline{d}_1}(x_1,\epsilon) \times \cdots \times B_{\overline{d}_N}(x_N,\epsilon) \times X_{N+1} \times X_{N+2} \times \cdots.
  \end{equation*}
  We claim $V \subseteq B_D(\mathbf{x},\epsilon) \subseteq U$. Given $\mathbf{y} \in \prod X_i$, $\overline{d}_i(x_i,y_i)/i \le 1/N$ for $i \ge N$. Therefore,
  \begin{equation*}
    D(\mathbf{x},\mathbf{y}) \le \max\left\{ \frac{\overline{d}_1(x_1,z_1)}{1}, \cdots, \frac{\overline{d}_N(x_N,z_N)}{N}, \frac{1}{N} \right\}.
  \end{equation*}
  If $\mathbf{y} \in V$, this expression is less than $\epsilon$, so $V \subseteq B_D(\mathbf{x},\epsilon)$ as desired, and the product topology is finer than the metric topology.
  \par Conversely, let $U = \prod U_i$ where $U_i$ is open in $X_i$ for
  $\alpha_1,\ldots,\alpha_n$ and $U_i = X_i$ otherwise. Let $\mathbf{x} \in U$
  be given, and choose $B_{\overline{d}_i}(x_i,\epsilon_i) \subseteq X_i$ for $i
  = \alpha_1,\ldots,\alpha_n$, where each $\epsilon_i \le 1$. Then, defining
  $\epsilon = \min\{\epsilon_i/i \mid i = \alpha_1,\ldots,\alpha_n\}$, we claim that $\mathbf{x} \in B_D(\mathbf{x},\epsilon) \subseteq U$. Let $\mathbf{y}$ be a point of $B_D(\mathbf{x},\epsilon)$. Then, for all $i$,
  \begin{equation*}
    \frac{\overline{d}_i(x_i,y_i)}{i} \le D(\mathbf{x},\mathbf{y}) < \epsilon.
  \end{equation*}
  Now if $i = \alpha_1,\ldots,\alpha_n$, then $\epsilon \le \epsilon_i/i$, so that $\overline{d}_i(x_i,y_i) < \epsilon_i \le 1$; it follows that $|x_i-y_i| < \epsilon_i$, and so $\mathbf{y} \in \prod U_i$ as desired. We thus have that the metric topology is finer than the product topology; combined with the above this implies the topologies are equal.
\end{proof}

\subsection{The Quotient Topology}
\setcounter{subsubsection}{1}
\begin{problem}\mbox{}
  \begin{enuma}
    \item Let $p\colon X \to Y$ be a continuous map. If there is a continuous map
      $f\colon Y \to X$ such that $p \circ f$ equals the identity map of $Y$, then
      $p$ is a quotient map.
    \item If $A \subset X$, a \emph{\textbf{retraction}} of $X$ onto $A$ is a
      continuous map $r \colon X \to A$ such that $r(a) = a$ for each $a \in A$.
      Show that a retraction is a quotient map.
  \end{enuma}
\end{problem}
\begin{proof}[Proof of $(a)$]
  If $V \subseteq Y$ with $U = p^{-1}(V) \subseteq X$ open, $f^{-1}(U) = f^{-1}(p^{-1}(V)) = (p \circ f)^{-1}(V) = V$ is open. Thus, $p$ is a quotient map.
\end{proof}
\begin{proof}[Proof of $(b)$]
  Let $\iota \colon A \to X$ be the inclusion map; then, $r \circ \iota$ is the
  identity on $A$, hence $r$ is a quotient map by $(a)$.
\end{proof}

\setcounter{subsubsection}{3}
\begin{problem}\mbox{}
  \begin{enuma}
  \item Define an equivalence relation on the plane $X = \mathbb{R}^2$ as follows:
    \begin{equation*}
      x_0 \times y_0 \sim x_1 \times y_1 \quad \text{if}~x_0 + y_0^2 = x_1 + y_1^2.
    \end{equation*}
    Let $X^*$ be the corresponding quotient space. It is homeomorphic to a familiar space; what is it?
  \item Repeat $(a)$ for the equivalence relation
    \begin{equation*}
      x_0 \times y_0 \sim x_1 \times y_1 \quad \text{if}~x_0^2 + y_0^2 = x_1^2 + y_1^2.
    \end{equation*}
  \end{enuma}
\end{problem}
\begin{proof}[Solution for $(a)$]
  Set $g(x \times y) = x + y^2 \in \mathbb{R}$. We see it is a surjection onto $\mathbb{R}$ since $\mathbb{R} \times \{0\} \mapsto \mathbb{R}$. It is continuous since for $x_0 \times y_0 \in X$, given $\epsilon > 0$, letting $\delta = \min(1,\epsilon/2(|y_0|+1))$, $\rho(x_0 \times y_0,x \times y) < \delta$ implies
  \begin{align*}
    |g(x_0 \times y_0)-g(x \times y)| &= |(x_0+y_0^2)-(x+y^2)|\\
    &\le |x_0-x|+|y_0^2-y^2|\\
    &\le |x_0-x|+|y_0+y||y_0-y|\\
    &\le |x_0-x|+|y-y_0|^2+2|y_0||y_0-y|\\
    &< 2(|y_0|+1)\delta < \epsilon.
  \end{align*}
  If we define $f\colon \mathbb{R} \to X$ by $x \mapsto x \times 0$, which is
  continuous since $(a,b) \times (c,d)$ maps back to $(a,b)$ which is open in
  $\mathbb{R}$, we see $g \circ f$ is the identity on $\mathbb{R}$, and so $g$
  is a quotient map by the lemma above. Since $x_0 \times y_0 \sim x_1 \times
  y_1 \iff g(x_0 \times y_0) = g(x_1 \times y_1)$, by Corollary $22.3$, this
  induces a bijective continuous map $g'\colon X^* \to \mathbb{R}$, which is a homeomorphism since $g$ was a quotient map.
\end{proof}
\begin{proof}[Solution for $(b)$]
  Set $g(x \times y) = x^2 + y^2 \in \mathbb{R}$. We see it is a surjection onto $\mathbb{R}_{\ge0}$, since $\mathbb{R} \times \{0\} \mapsto \mathbb{R}_{\ge 0}$, and it does not map to anything else since $x^2 + y^2 \ge 0$ for all $x,y$. It is continuous since for $x_0 \times y_0 \in X$, given $\epsilon > 0$, letting $\delta = \min(1,\epsilon/2(|y_0|+|x_0|+1))$, $\rho(x_0\times y_0,x \times y) < \delta$ implies
  \begin{align*}
    |g(x_0\times y_0) - g(x\times y)| &= |(x_0^2+y_0^2)-(x^2+y^2)|\\
    &\le |x_0^2-x^2| + |y_0^2-y^2|\\
    &\le |x_0-x||x-x_0+2x_0| + |y_0-y||y-y_0+2y_0|\\
    &\le |x_0-x|(1+2|x_0|) + |y_0-y|(1+2|y_0|)\\
    &\le 2\delta(|x_0| + |y_0| + 1) < \epsilon.
  \end{align*}
  We define $f\colon \mathbb{R}_{\ge 0} \to X$ by $x \mapsto \sqrt{x} \times 0$,
  which is continuous since the preimage of $(a,b) \times (c,d)$, if $(c,d) \ni
  0$, is the open set $\mathbb{R}_{\ge 0} \cap (a',b')$, where $a' = a^2$ if $a
  \ge 0$, and $-1$ otherwise, and similarly for $b'$ (we chose $-1$ out of
  convenience; we really only have to make sure the preimage is a half-open set
  $[0,b')$ or the empty set in these cases); if $(c,d) \not\ni 0$, then the
    preimage would be empty. We then see $g \circ f$ is the identity on
    $\mathbb{R}_{\ge0}$, and so $g$ is a quotient map by the lemma above. Since
    $x_0 \times y_0 \sim x_1 \times y_1 \iff g(x_0 \times y_0) = g(x_1 \times
  y_1)$, by Corollary $22.3$, this induces a bijective continuous map $g'\colon X^* \to \mathbb{R}_{\ge 0}$, which is a homeomorphism since $g$ was a quotient map.
\end{proof}

\setcounter{subsubsection}{5}
\begin{problem}
  Recall that $\mathbb{R}_K$ denotes the real line in the $K$-topology. Let $Y$
  be the quotient space obtained by $\mathbb{R}_K$ by collapsing the set $K$ to
  a point; let $p\colon \mathbb{R}_K \to Y$ be the quotient map.
  \begin{enuma}
    \item Show that $Y$ satisfies the $T_1$ axiom, but is not Hausdorff.
    \item Show that $p \times p\colon \mathbb{R}_K \times \mathbb{R}_K \to Y \times Y$ is not a quotient map.
  \end{enuma}
\end{problem}
\begin{proof}[Proof of $(a)$]
  Recall by p.~141 that it suffices to show every element in the partition, i.e., one-point sets $\{x\}$ for $x \notin K$ and $K$ itself, are closed in $\mathbb{R}_K$. The former are closed since $\mathbb{R}_K$ is $T_1$ since it is Hausdorff by Example 31.1, and the latter is closed since it is the complement of $\mathbb{R} \setminus K$. Thus, $Y$ is $T_1$.
  \par We now show $Y$ is not Hausdorff. We claim that $p(0),p(K)$ are not separable; note $p(0) \ne p(K)$ since they are in different equivalence classes. Suppose $Y$ is Hausdorff, and let $V_1 \ni p(0),V_2 \ni p(K)$ be a separation in $Y$; they have open preimages $U_1 = p^{-1}(V_1) \ni 0, U_2 = p^{-1}(V_2) \supseteq K$ by definition of a quotient map. There then exists $(a,b) \setminus K \ni 0$ contained in $U_1$, and choosing $n \in \mathbb{N}$ such that $1/n < b$, there exists $(c,d) \ni 1/n$ contained in $U_2$, where we can assume $1/(n+1) \le c$, since if not, we can take the intersection with $(1/(n+1),d)$. Then, $(c,1/n) \subseteq U_1 \cap U_2$, and so $p((c,1/n)) \subseteq V_1 \cap V_2$, which is a contradiction, and so $Y$ is not Hausdorff.
\end{proof}
\begin{proof}[Proof of $(b)$]
  By Exercise \ref{exc:17.13}, we see that since $Y$ is not Hausdorff by $(a)$,
  the diagonal $\Delta_Y \subseteq Y \times Y$ is not closed. $(p^{-1} \times
  p^{-1})(\Delta_Y) = \Delta_K \cup (K \times K)$, where $\Delta_K \subseteq
  \mathbb{R}_K \times \mathbb{R}_K$ is the diagonal in $\mathbb{R}_K$. However,
  $\Delta_K$ is closed by Exercise \ref{exc:17.13} since $\mathbb{R}_K$ is Hausdorff by Example 31.1, and so $\Delta_K \cup (K \times K)$ is closed since is is the finite union of closed sets. Thus, the inverse image of the non-closed set $\Delta_Y$ is closed, and so $p \times p$ is not a quotient map.
\end{proof}

\section{Connectedness and Compactness}
\setcounter{subsection}{22}
\subsection{Connected Spaces}
\setcounter{subsubsection}{7}
\begin{problem}\label{exc:23.8}
  Determine whether or not $\mathbb{R}^\omega$ is connected in the uniform topology.
\end{problem}
\begin{proof}[Solution]
  Let $\mathbb{R}^\omega = A \cup B$, where $A$ is the set of bounded sequences and $B$ is the set of unbounded sequences of reals. $A,B$ are disjoint, and so it remains to show they are open. Suppose $a = (a_1,a_2,\ldots) \in A$ and $b = (b_1,b_2,\ldots) \in B$. Since $|a_i| < N$ for all $i$ for some $N$, and since $|b_i| > N+1$ for all $i$ larger than some $I$, we have that $\overline{d}(a_i,b_i) = 1$ for all $i \ge I$. Thus, $\overline{\rho}(a,b) = 1$ for any $a \in A, b\in B$, and so the open balls with radius $1/2$ around $a,b$ are fully contained in $A,B$ respectively.
\end{proof}

\setcounter{subsubsection}{10}
\begin{problem}
  Let $p\colon X \to Y$ be a quotient map. Show that if each set $p^{-1}(\{y\})$ is connected, and if $Y$ is connected, then $X$ is connected.
\end{problem}
\begin{proof}
  Suppose not. Then, $X = A \cup B$ for $A,B$ open, disjoint sets. Consider $C =
  \{y \in Y \mid p^{-1}(\{y\}) \subseteq A\}, D = \{y \in Y \mid p^{-1}(\{y\})
\subseteq B\}$; we see that these sets are such that $C \cup D = Y$ since $p^{-1}(\{y\})$ connected implies it is in either $A$ or $B$ by Lemma $23.2$. $C,D$ are then disjoint by definition and $p^{-1}(C) = A, p^{-1}(D) = B$ by the fact that $p$ is surjective. $p$ quotient map implies that $C,D$ are then open, and so $Y = C \cup D$ is a separation, a contradiction.
\end{proof}

\subsection{Connected Subspaces of the Real Line}
\setcounter{subsubsection}{6}
\begin{problem}\mbox{}
  \begin{enuma}
    \item Let $X$ and $Y$ be ordered sets in the order topology. Show that if
      $f\colon X \to Y$ is order preserving and surjective, then $f$ is a homeomorphism.
    \item Let $X = Y = \overline{\mathbb{R}}_+$. Given a positive integer $n$, show that the function $f(x) = x^n$ is order preserving and surjective. Conclude that its inverse, the \emph{$n$th root function,} is continuous.
    \item Let $X$ be the subspace $(-\infty,-1)\cup[0,\infty)$ of $\mathbb{R}$.
      Show that the function $f\colon X\to\mathbb{R}$ defined by setting $f(x) = x + 1$ if $x < -1$, and $f(x) = x$ if $x \ge 0$, is order preserving and surjective. Is $f$ a homeomorphism? Compare with $(a)$.
  \end{enuma}
\end{problem}
\begin{proof}[Proof of $(a)$]
  $f$ is injective since if $f(a) = f(b)$ but $a \ne b$, then (with possible swapping) $a<b$, and so $f(a) < f(b)$, a contradiction. We thus must show $f$ and $f^{-1}$ are continuous. But $f$ is continuous since $f^{-1}((a,b)) = (f^{-1}(a),f^{-1}(b))$ is open (apply the same argument to the intervals of the form $[a_0,b),(a,b_0]$ for $a_0,b_0$ minimal and maximal respectively); the same argument applies for $f^{-1}$ as well.
\end{proof}
\begin{proof}[Proof of $(b)$]
  $f(x) = x^n$ is order preserving since $a < b \implies a/b < 1 \implies a^n/b^n < 1 \implies a^n < b^n \implies f(a) < f(b)$. $f$ is continuous since it is the product of $n$ copies of the identity function, which is continuous. We want to show $f$ is surjective. Letting $N = \{x^n \mid x \in \mathbb{Z}_{\ge 0}\}$, we see that every real number $y \in Y$ is between two consecutive members of $N$, or it is already an $n$th power of an integer, in which case it is trivially mapped to by its $n$th root. In the case $y \in Y$ is not an $n$th power, we have $f(n) < y < f(n+1)$, and so by the Intermediate value theorem (Theorem 24.3), we see that there exists a point $c \in X$ such that $f(c) = r$, i.e., $f$ is surjective.
  \par Since $f$ is order preserving and surjective, by $(a)$ it is then a homeomorphism, and so $f^{-1}$, the $n$th root function, is also continuous.
\end{proof}
\begin{proof}[Proof of $(c)$]
  $f$ is order-preserving on $(-\infty,-1)$ since $a < b \implies f(a) = a+1 < b+1 = f(b)$, and on $[0,\infty)$ since it is the identity. We check that it is order preserving around the boundary. So, suppose $a < -1$ and $b \ge 0$. Then, $a < b$ but also $a+1 < 0 \le b$, and so $f$ is order-preserving. $f$ is surjective since if $x \in \mathbb{R}$, if $x \ge 0$ its preimage is itself, and if $x < 0$, its preimage is $x-1$. $f$ is not a homeomorphism by Theorem $23.6$ since $\mathbb{R}$ is connected but $X$ is not, by considering $f^{-1}(\mathbb{R})$.
  \par This does not contradict $(a)$ since $X$ is not in the order topology. Even if $\mathbb{R}$ is in the order topology, the subspace topology induced on $X$ is not the order topology. For, $(-1/2,1)$ is open in $\mathbb{R}$, and so $(-1/2,1) \cap X = [0,1)$ is open in $X$, but not open in the order topology on $X$.
\end{proof}

\begin{problem}\label{exc:24.8}\mbox{}
  \begin{enuma}
    \item Is a product of path-connected spaces necessarily path connected?
    \item If $A \subset X$ and $A$ is path connected, is $\overline{A}$ necessarily path connected?
    \item If $f\colon X \to Y$ is continuous and $X$ is path connected, is $f(X)$ necessarily path connected?
    \item If $\{A_\alpha\}$ is a collection of path-connected subspaces of $X$ and if $\bigcap A_\alpha \ne \emptyset$, is $\bigcup A_\alpha$ necessarily path connected?
  \end{enuma}
\end{problem}
\begin{proof}[Solution for $(a)$]
  Yes. Let $X = \prod X_\alpha$, $\mathbf{x},\mathbf{y} \in X$. Since each
  $X_\alpha$ is path connected, we have $f_\alpha\colon [0,1] \to X_\alpha$ continuous such that $f_\alpha(0) = x_\alpha,f_\alpha(1) = y_\alpha$, where we assume the closed interval is $[0,1]$ after composition with multiplication and addition, which are continuous operations. Thus we have the function $\mathbf{f} = (f_\alpha)$, which is continuous by Theorem $19.6$, with $\mathbf{f}(0) = \mathbf{x},\mathbf{f}(1) = \mathbf{y}$, and so $X$ is path-connected.
\end{proof}
\begin{proof}[Solution for $(b)$]
  No, since $\overline{S}$ in Example 24.7 is not path-connected while $S$ is:
  it is the image of the continuous map $x \mapsto (x,\sin(1/x))$ from
  $\mathbb{R}_{>0}$ to $\mathbb{R}^2$.
\end{proof}
\begin{proof}[Solution for $(c)$]
  Yes. For, let $x,y \in f(X)$, and choose $x_0 \in f^{-1}(x), y_0 \in
  f^{-1}(y)$. Then, there exists continuous $g\colon [a,b] \to X$ such that
  $g(a) = x_0,g(b) = y_0$, and so its composition $f \circ g\colon [a,b] \to Y$ is continuous with $(f \circ g)(a) = x, (f \circ g)(b) = y$.
\end{proof}
\begin{proof}[Solution for $(d)$]
  Yes. Let $x,y \in \bigcup A_\alpha$ and $p \in \bigcap A_\alpha$. Then, there
  exists a continuous map $f\colon [a,b] \to \bigcup A_\alpha$ with $f(a) = x,
  f(b) = p$, and similarly $g\colon [b,c] \to \bigcup A_\alpha$ with $f(b) = p, f(c) = y$, since $a,p \in A_\alpha$ for some $\alpha$ and similarly for $y$ (we are free to have $\Dom g = [b,c]$ by composition with multiplication and addition, which are continuous). Then, by the pasting lemma (Theorem 18.3) since $f,g$ are continuous and $f(b) = g(b)$, we see that $h=f$ on $[a,b]$ and $h=g$ on $[b,c]$ is a continuous map such that $h(a) = x, h(c) = y$.
\end{proof}

\subsection{Components and Local Connectedness}
\setcounter{subsubsection}{1}
\begin{problem}\mbox{}
  \begin{enuma}
    \item What are the components and path components of $\mathbb{R}^\omega$ (in the product topology)?
    \item Consider $\mathbb{R}^\omega$ in the uniform topology. Show that $\mathbf{x}$ and $\mathbf{y}$ lie in the same component of $\mathbb{R}^\omega$ if and only if the sequence
      \begin{equation*}
        \mathbf{x} - \mathbf{y} = (x_1-y_1,x_2-y_2,\ldots)
      \end{equation*}
      is bounded.
    \item Give $\mathbb{R}^\omega$ the box topology. Show that $\mathbf{x}$ and $\mathbf{y}$ lie in the same component of $\mathbb{R}^\omega$ if and only if the sequence $\mathbf{x}-\mathbf{y}$ is ``eventually zero.''
  \end{enuma}
\end{problem}
\begin{proof}[Solution for $(a)$]
  By Exercise $\ref{exc:24.8}(a)$, $\mathbb{R}^\omega$ is path connected, for Theorem $19.6$ is not limited to finite product topologies. Thus, $\mathbb{R}^\omega$ is the only path component, and so $\mathbb{R}^\omega$ is the only component as well since path connected $\implies$ connected.
\end{proof}
\begin{proof}[Proof of $(b)$]
  We first define $\varphi\colon \mathbf{x} \mapsto \mathbf{x} - \mathbf{y}$. We
  recall that since $\overline{\rho}(\varphi(\mathbf{x}),\varphi(\mathbf{z})) =
  \overline{\rho}(\mathbf{x},\mathbf{z})$, by Exercise \ref{exc:21.2} $\varphi$ is an isometric imbedding that is moreover surjective (the preimage of any $\mathbf{z}$ is $\mathbf{z}+\mathbf{y}$), $\varphi$ is a homeomorphism. Thus, $\mathbf{x}-\mathbf{y}$ is in the same component as $0$ if and only if $\mathbf{x}$ is in the same component as $\mathbf{y}$, for $\varphi,\varphi^{-1}$ do not modify the topology of $\mathbb{R}^\omega$.
  \par It therefore suffices to check the case $\mathbf{y} = 0$. Suppose
  $\mathbf{x}$ is bounded; then, we define $f\colon [0,1] \to \mathbb{R}^\omega$
  where $f(t) = (x_1t,x_2t,\ldots)$. This is continuous since given $\epsilon >
  0$, $B(f(t),\epsilon) \supseteq f(B(t,\epsilon/\sup\{\lvert x_n\rvert\}))$, where $\sup\{\lvert x_n\rvert\} < \infty$ by boundedness of $\mathbf{x}$. Thus, $f$ connects $0$ and $\mathbf{x}$, i.e., they are in the same path component, and therefore the same component by Theorem 25.5.
  \par Conversely, recall by Exercise \ref{exc:23.8} that we have the separation $\mathbb{R}^\omega = A \cup B$, where $A$ is the set of bounded sequences and $B$ is the set of unbounded sequences of reals. If $\mathbf{x}$ is unbounded it is in $B$ and so is not in the same component as $0$.
\end{proof}
\begin{proof}[Proof of $(c)$]
  $\mathbf{x}$ ``eventually zero'' here means that $x_i = 0$ for all $i \ge N$ for some $N$. Note by the same argument as in $(b)$, it suffices to consider the case $\mathbf{y} = 0$.
  \par Suppose first that $\mathbf{x}$ is not eventually zero. Define the
  function $\mathbf{f} = (f_n)$, where $f_n(a) = na/|x_n|$ if $x_n \ne 0$, and
  $a$ otherwise. $\mathbf{f}$ is continuous since each $f_n$ is continuous since
  it is linear, and so if $f_n^{-1}(U_n) = V_n$, we have $\mathbf{f}^{-1}(\prod
  U_n) = \prod V_n$. Note that this is a bijection since each component has an
  inverse $f_n^{-1}(a) = |x_n|a/n$ if $x_n \ne 0$, and $a$ otherwise, and
  moreover since the inverse is continuous since each component is linear, we
  have a homeomorphism $\mathbf{f}\colon \mathbb{R}^\omega \to \mathbb{R}^\omega$. Since there are infinitely many $n$ such that $x_n \ne 0$, and so infinitely many $n$ such that $f_n(x_n) = n$, we have that $\mathbf{f}(\mathbf{x})$ is unbounded, and thus, by the separation of $\mathbb{R}^\omega$ in the box topology in Example 23.6, we have that $\mathbf{f}(\mathbf{x})$ and $0$ are in different components. Since $\mathbf{f}$ is a homeomorphism, this implies $\mathbf{x}$ and $0$ are in different components as well.
  \par Conversely, suppose $\mathbf{x}$ is eventually zero. Then, $x_n = 0$ for
  all $n \ge N$ for some $N$, and so $\mathbf{x} \in \mathbb{R}^N \times \{0\}
  \times \{0\} \times \cdots \subseteq \mathbb{R}^\omega$; this subspace is
  homeomorphic to $\mathbb{R}^N$. Since $\mathbb{R}^N$ is connected by Theorem $23.6$, we see that $\mathbf{x}$ and $0$ are in the same component.
\end{proof}

\setcounter{subsection}{26}
\subsection{Compact Subspaces of the Real Line}
\setcounter{subsubsection}{3}
\begin{problem}
  Show that a connected metric space having more than one point is uncountable.
\end{problem}
\begin{proof}
  Let $X$ be a connected metric space with the metric $d$, and let $x_0,x_1 \in
  X$ be distinct. Let $d(x_0,x_1) = r$, and define $f(x) = d(x_0,x)$. $f$ is
  continuous by the discussion on p.~175. We see that $f(x_0) = 0, f(x) = r$, and so by the intermediate value theorem (Theorem 24.3), $f(X) \supseteq [0,r]$, i.e., $f$ maps onto $[0,r]$.
  \par Now suppose $X$ is countable. Then, by Theorem $7.1$ there exists a
  surjective function $g\colon \mathbb{Z}_+ \to X$, and so $f \circ g\colon \mathbb{Z}_+ \to f(X)$ maps onto $[0,r]$, which is a contradiction since $[0,r]$ is uncountable by Corollary $27.8$.
\end{proof}

\setcounter{subsection}{28}
\subsection{Local Compactness}
\setcounter{subsubsection}{3}
\begin{problem}
  Show that $[0,1]^\omega$ is not locally compact in the uniform topology.
\end{problem}
\begin{proof}
  Suppose $X = [0,1]^\omega$ is locally compact, and in particular at $0$. Then, there exists $C$ compact that contains a neighborhood $U \ni 0$. There exists $X \cap B_{\overline{\rho}}(0,\epsilon) \subseteq U$; we see that $\{0,\epsilon/3\}^\omega \subseteq X \cap B_{\overline{\rho}}(0,\epsilon)$. $\{0,\epsilon/3\}^\omega$ is closed since
  \begin{equation*}
    \{0,\epsilon/3\}^\omega = \prod \{0,\epsilon/3\} = \prod \overline{\{0,\epsilon/3\}} = \overline{\prod \{0,\epsilon/3\}} = \overline{\{0,\epsilon/3\}^\omega}
  \end{equation*}
  in the product topology by Theorem $19.5$, which is finer than the uniform topology, and so it is compact by Theorem $26.2$ since it is a closed subset of $C$ compact, i.e., limit point compact by Theorem $28.2$.
  \par We claim this is a contradiction. Consider $\mathbf{x} \in X$, and the ball $X \cap B_{\overline{\rho}}(\mathbf{x},\epsilon/9)$. Note that the distance between any two distinct points of $\{0,\epsilon/3\}^\omega$ is $\epsilon/3$, and so since the diameter of $X \cap B_{\overline{\rho}}(\mathbf{x},\epsilon/9)$ is at most $2\epsilon/9$, $X \cap B_{\overline{\rho}}(\mathbf{x},\epsilon/9)$ contains at most one point of $\{0,\epsilon/3\}^\omega$. Thus, $\{0,\epsilon/3\}^\omega$ contains no limit points, and so is not limit point compact, a contradiction.
\end{proof}

\setcounter{subsubsection}{7}
\begin{problem}
  Show that the one-point compactification of $\mathbb{Z}_+$ is homeomorphic to the subspace $\{0\} \cup \{1/n \mid n \in \mathbb{Z}_+\}$ of $\mathbb{R}$.
\end{problem}
\begin{proof}
  Let $K = \{1/n \mid n \in \mathbb{Z}_+\}$. Let $f\colon \mathbb{R}_+ \to
  \mathbb{R}_+$ such that $f(x) = 1/x$; this is a homeomorphism since it is
  continuous and is its own inverse. By Theorem $18.2(d)$ and $18.2(e)$,
  $f\colon \mathbb{Z}_+ \to f(\mathbb{Z}_+) = K$ is continuous, and again is a
  homeomorphism since it is its own inverse. Now consider $Y = \{0\} \cup K$,
  which is closed and bounded and therefore compact by Theorem $27.3$, and
  Hausdorff by Theorem $17.11$. Since $K' = \{0\}$ by Example 17.8, we know
  $Y$ is the one-point compactification of $K$. If $X = \{p\} \cup \mathbb{Z}_+$
  is the one-point compactification of $\mathbb{Z}_+$, and letting $g\colon p
  \mapsto 0 \in Y$, which is clearly continuous, the function $h\colon X \to Y$ defined by the pasting lemma (Theorem $18.3$) applied to $f,g$ is also continuous, and has continuous inverse defined by the pasting lemma applied to $f^{-1},g^{-1}$, and so is a homeomorphism $X \leftrightarrow Y$.
\end{proof}

\section{Countability and Separation Axioms}
\setcounter{subsection}{29}
\subsection{The Countability Axioms}
\setcounter{subsubsection}{4}
\begin{problem}\label{exc:30.5}\mbox{}
  \begin{enuma}
    \item Show that every metrizable space with a countable dense subset has a
      countable basis.
    \item Show that every metrizable Lindel\"of space has a countable basis.
  \end{enuma}
\end{problem}
\begin{proof}[Proof of $(a)$]
  Let $X$ be a metrizable space and $A$ a countable dense subset. We claim that
  the set of open balls in $X$ below is a basis for $X$:
  \begin{equation*}
    \mathcal{B} \coloneqq \{ B(a,1/n) \subseteq X \mid a \in A,\ n \in
      \mathbb{N} \}.
  \end{equation*}
  Note $\mathcal{B}$ is countable since is in bijection with
  $A \times \mathbb{N}$. So let $x$ be contained in an open subset $U \subseteq
  X$; since $X$ is metrizable, $x \in B(x,\epsilon) \subseteq U$ for some small
  $\epsilon$. Let $n$ be such that $1/n < \epsilon/2$. Then,
  since $A$ is dense, some $a \in A$ is contained in $B(x,1/n)$, and conversely
  $x \in B(a,1/n)$. By the triangle inequality, $x \in B(a,1/n) \subseteq U$, so
  by Lemma $13.2$ we are done.
\end{proof}
\begin{proof}[Proof of $(b)$]
  Let $X$ be a metrizable space. Then, the set of open balls 
  \begin{equation*}
    \widetilde{\mathcal{B}}_n \coloneqq \{ B(x,1/n) \subseteq X \mid x \in X \}
  \end{equation*}
  is an open cover of $X$ for each $n \in \mathbb{N}$; since $X$ is Lindel\"of,
  it has a countable subcover $\mathcal{B}_n$. We claim $\mathcal{B} \coloneqq
  \bigcup_{n \in \mathbb{N}} \mathcal{B}_n$ is a basis for $X$; note it is
  countable since it is a countable union of countable sets. So let $x \in
  B(x,\epsilon) \subseteq U$ as before, and let $n$ such that $1/n <
  \epsilon/2$. Then, there is some $x' \in X$ such that 
  $B(x',1/n) \in \mathcal{B}_n$ contains $x$. By the triangle inequality,
  $x \in B(x',1/n) \subseteq U$, so by Lemma $13.2$ we are done.
\end{proof}

\setcounter{subsubsection}{7}
\begin{problem}
  Which of our four countability axioms does $\mathbb{R}^\omega$ in the uniform topology satisfy?
\end{problem}
\begin{proof}[Solution]
  $\mathbb{R}^\omega$ is first countable since it is metrizable (see p.~130 and
  Example 30.2), but is not second countable by Example 30.2. By Exercise
  \ref{exc:30.5}, we then see that $\mathbb{R}^\omega$ does not have a countable dense subset, and is also not Lindel\"of.
\end{proof}

\begin{problem}
  Let $A$ be a closed subspace of $X$. Show that if $X$ is Lindel\"of, then $A$ is Lindel\"of. Show by example that if $X$ has a countable dense subset, $A$ need not have a countable dense subset.
\end{problem}
\begin{proof}
  $X$ is Lindel\"of if and only if a collection of closed subsets of $X$ with empty intersection has a countable subcollection with empty intersection by taking complements in Theorem $30.3(a)$. Now suppose we have a collection $\mathcal{C}$ of closed subsets of $A$ with empty intersection; it is then also a collection of closed subsets of $X$ with empty intersection by Theorem $17.3$ since $A$ is closed, and so has a countable subcollection with empty intersection since $X$ is Lindel\"of. Thus, $A$ is also Lindel\"of.
  \par Now let $X = \mathbb{R}_\ell^2$. We see $\mathbb{Q}^2$ is countable, and is dense in $X$ since if we take $x \in X$ and a neighborhood $U \ni x$, there exists a basis element $[a,b) \times [c,d) \subseteq U$ containing $x$, which intersects $\mathbb{Q}^2$ by the fact that $(a,b) \times (c,d) \cap \mathbb{Q}^2 \ne \emptyset$ since $\mathbb{Q}$ is dense in $\mathbb{R}$. Thus, $X$ has a countable dense subset; we claim that $L = \{x \times (-x) \mid x \in \mathbb{R}_\ell\}$ is a closed subspace of $X$ that does not have a countable dense subset. $L$ is closed since if $(x_1,x_2) \in X \setminus L$, then letting $d = x_1+x_2$, the basis element $[x_1-d/3,x_1+d/3),[x_2-d/3,x_2+d/3)$ does not intersect $L$. But then, $L$ has the discrete topology since $\{(x,-x)\} = L \cap [x,b) \times [-x,d)$ is open in $L$. Thus, if $A \subseteq L$, $A = \overline{A}$ by discreteness, and so $\overline{A} = L$ is true if and only if $A = L$. Thus, $L$ has no countable dense subset.
\end{proof}

\setcounter{subsubsection}{16}
\begin{problem}
  Give $\mathbb{R}^\omega$ the box topology. Let $\mathbb{Q}^\infty$ denote the subspace consisting of sequences of rationals that end in an infinite string of $0$'s. Which of our four countability axioms does this space satisfy?
\end{problem}
\begin{proof}
  We claim $\mathbb{Q}^\infty$ is not first countable, and therefore not second
  countable. Suppose we have a countable basis $\{U_i\}$ at $\mathbf{0} =
  (0,0,0,\ldots) \in \mathbb{Q}^\omega$. Let $V_j \subsetneq \pi_j(U_j)$ open in
  $\mathbb{Q}$ with the subspace topology induced by $\mathbb{R}$.
  Then, the neighborhood $\prod_j V_j \ni \mathbf{0}$ does
  not contain any $U_i$, so $\{U_i\}$ is not a basis and
  $\mathbb{Q}^\infty$ is not first or second countable.
  \par We now show $\mathbb{Q}^\infty$ has a countable dense subset. For, $Q^n = \mathbb{Q}^n \times \{0\} \times \{0\}$ are countable since they are finite products of countable sets, and so their countable union $\mathbb{Q}^\infty = \bigcup Q^n$ is also countable. Thus, $\mathbb{Q}^\infty$ is countable and so is a countable dense subset of itself.
  \par We now show $\mathbb{Q}^\infty$ is Lindel\"of. Suppose $\mathcal{V}$ is an open covering of $\mathbb{Q}^\infty$. Then, since $\mathbb{Q}^\infty$ is countable, choosing for every $\mathbf{x} \in \mathbb{Q}^\infty$ one element $V \in \mathcal{V}$ such that $\mathbf{x} \in V$, we get a countable subcover of $\mathbb{Q}^\infty$.
\end{proof}

\subsection{The Separation Axioms}
\setcounter{subsubsection}{2}
\begin{problem}
  Show that every order topology is regular.
\end{problem}
\begin{proof}
  Let $X$ be an ordered set with the order topology. $X$ is Hausdorff and
  therefore $T_1$ by Theorem $17.11$. It therefore suffices to show the
  condition in Lemma $31.1(a)$. So let $x \in X$ and let $U$ be an open
  neighborhood of $x$; we will construct a basis element $V$ of the order
  topology such that $x \in V$ and $\overline{V} \subseteq U$.
  \par Suppose $x$ is neither the smallest nor largest element of $X$. Then, $x
  \in (a,b) \subseteq U$ for some basis element $(a,b)$ of the order topology.
  If $(a,x),(x,b)$ are nonempty, then let $V = (u,v)$ where $u \in (a,x)$ and
  $v \in (x,b)$. If $(a,x) = \emptyset$, let $u=a$, so that $V = (u,v) = [x,v)$; if
  $(x,b) = \emptyset$, let $v=b$, so that $V = (u,v) = (u,x]$.
  Then, $x \in (u,v)$ and $\overline{(u,v)} \subseteq (a,b) \subseteq U$.
  \par Now suppose $x$ is the smallest (resp.~largest) element of $X$. Then, $x
  \in [x,b) \subseteq U$ (resp.~$x \in (a,x] \subseteq U$) for some basis
  element $[x,b)$ (resp.~$(a,x]$) of the order topology. Now if $(x,b)$
  (resp.~$(a,x)$) is nonempty, then let $V = [x,v)$ (resp.~$(u,x]$) where
  $v \in (x,b)$ (resp.~$u \in (a,x)$). On the other hand, if $(x,b)$
  (resp.~$(a,x)$) is empty, then let $v=b$ (resp.~$u=a$), so that $V = \{x\}$.
  We then have $x \in V$ and $\overline{V} \subseteq U$.
  \par If $x$ is the smallest and largest element of $X$, then $X = \{x\}$ is
  trivially regular.
\end{proof}

\subsection{Normal Spaces}

\cleardoublepage
\pdfbookmark[1]{List of Solved Exercises}{det}
{\footnotesize\tableofcontents}
\end{document}
