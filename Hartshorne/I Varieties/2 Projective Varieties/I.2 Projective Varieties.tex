\documentclass[12pt,letterpaper]{article}
\usepackage{geometry}
\geometry{letterpaper}
\usepackage{amsmath,amssymb,amsthm,mathrsfs}
\usepackage{mathtools}
\usepackage{ifpdf}
  \ifpdf
    \setlength{\pdfpagewidth}{8.5in}
    \setlength{\pdfpageheight}{11in}
  \else
\fi
\usepackage{hyperref}

\usepackage{tikz}
\usepackage{tikz-cd}
\usetikzlibrary{decorations.markings}
\tikzset{
  open/.style = {decoration = {markings, mark = at position 0.5 with { \node[transform shape] {\tikz\draw[fill=white] (0,0) circle (.3ex);}; } }, postaction = {decorate} },
  closed/.style = {decoration = {markings, mark = at position 0.5 with { \node[transform shape, xscale = .8, yscale=.4] {\upshape{/}}; } }, postaction = {decorate} },
  imm/.style = {decoration = {markings, mark = at position 0.3 with { \node[transform shape, xscale = .8, yscale=.4] {\upshape{/}}; }, mark = at position 0.6 with { \node[transform shape] {\tikz\draw[fill=white] (0,0) circle (.3ex);}; } }, postaction = {decorate} }
}

\usepackage{braket}

\usepackage{csquotes}
\usepackage[american]{babel}
\usepackage[style=alphabetic,firstinits=true,backend=biber,texencoding=utf8,bibencoding=utf8]{biblatex}
\bibliography{../../../References}
\AtEveryBibitem{\clearfield{url}}
\AtEveryBibitem{\clearfield{doi}}
\AtEveryBibitem{\clearfield{issn}}
\AtEveryBibitem{\clearfield{isbn}}
\renewbibmacro{in:}{}
\DeclareFieldFormat{postnote}{#1}
\DeclareFieldFormat{multipostnote}{#1}

\renewcommand{\theenumi}{$(\alph{enumi})$}
\renewcommand{\labelenumi}{\theenumi}

\newcounter{enumacounter}
\newenvironment{enuma}
{\begin{list}{$(\alph{enumacounter})$}{\usecounter{enumacounter} \parsep=0em \itemsep=0em \leftmargin=2.75em \labelwidth=1.5em \topsep=0em}}
{\end{list}}
\newcounter{enumicounter}
\newenvironment{enumi}
{\begin{list}{$(\roman{enumicounter})$}{\usecounter{enumicounter} \parsep=0em
\itemsep=0em \leftmargin=2.5em \labelwidth=1.75em \topsep=0em}}
{\end{list}}
\newcounter{enumdcounter}
\newenvironment{enumd}
{\begin{list}{$(\arabic{enumdcounter})$}{\usecounter{enumdcounter} \parsep=0em \itemsep=0em \leftmargin=1.75em \labelwidth=1.5em \topsep=0em}}
{\end{list}}
\newtheorem*{theorem}{Theorem}
\newtheorem*{universalproperty}{Universal Property}
\newtheorem{problem}{Exercise}[section]
\newtheorem{subproblem}{Problem}[problem]
\newtheorem{lemma}{Lemma}%[section]
\newtheorem{proposition}{Proposition}
\newtheorem{property}{Property}[problem]
\newtheorem*{lemma*}{Lemma}
\theoremstyle{definition}
\newtheorem*{definition}{Definition}
\newtheorem*{claim}{Claim}
\theoremstyle{remark}
\newtheorem*{remark}{Remark}

\numberwithin{equation}{section}
\numberwithin{figure}{problem}
\renewcommand{\theequation}{\arabic{section}.\arabic{equation}}

\DeclareMathOperator{\Ann}{Ann}
\DeclareMathOperator{\Ass}{Ass}
\DeclareMathOperator{\Supp}{Supp}
\DeclareMathOperator{\WeakAss}{\widetilde{Ass}}
\let\Im\relax
\DeclareMathOperator{\Im}{im}
\DeclareMathOperator{\Spec}{Spec}
\DeclareMathOperator{\SPEC}{\mathbf{Spec}}
\DeclareMathOperator{\Sp}{sp}
\DeclareMathOperator{\maxSpec}{maxSpec}
\DeclareMathOperator{\Hom}{Hom}
\DeclareMathOperator{\Soc}{Soc}
\DeclareMathOperator{\Ht}{ht}
\let\AA\relax
\DeclareMathOperator{\AA}{\mathbb{A}}
\DeclareMathOperator{\V}{\mathbf{V}}
\DeclareMathOperator{\Aut}{Aut}
\DeclareMathOperator{\Char}{char}
\DeclareMathOperator{\Frac}{Frac}
\DeclareMathOperator{\Proj}{Proj}
\DeclareMathOperator{\stimes}{\text{\footnotesize\textcircled{s}}}
\DeclareMathOperator{\End}{End}
\DeclareMathOperator{\Ker}{Ker}
\DeclareMathOperator{\Coker}{coker}
\DeclareMathOperator{\LCM}{LCM}
\DeclareMathOperator{\Div}{Div}
\DeclareMathOperator{\id}{id}
\DeclareMathOperator{\Cl}{Cl}
\DeclareMathOperator{\dv}{div}
\DeclareMathOperator{\Gr}{Gr}
\DeclareMathOperator{\pr}{pr}
\DeclareMathOperator{\trdeg}{trdeg}
\DeclareMathOperator{\rank}{rank}
\DeclareMathOperator{\codim}{codim}
\DeclareMathOperator{\sgn}{sgn}
\DeclareMathOperator{\GL}{GL}
\newcommand{\GR}{\mathbb{G}\mathrm{r}}
\newcommand{\gR}{\mathrm{Gr}}
\newcommand{\EE}{\mathscr{E}}
\newcommand{\FF}{\mathscr{F}}
\newcommand{\GG}{\mathscr{G}}
\newcommand{\HH}{\mathscr{H}}
\newcommand{\II}{\mathscr{I}}
\newcommand{\LL}{\mathscr{L}}
\newcommand{\MM}{\mathscr{M}}
\newcommand{\OO}{\mathcal{O}}
\newcommand{\Ss}{\mathscr{S}}
\newcommand{\Af}{\mathfrak{A}}
\newcommand{\Aa}{\mathscr{A}}
\newcommand{\PP}{\mathbb{P}}
\newcommand{\red}{\mathrm{red}}
\newcommand{\Sh}{\mathfrak{Sh}}
\newcommand{\Psh}{\mathfrak{Psh}}
\newcommand{\LRS}{\mathsf{LRS}}
\newcommand{\Sch}{\mathfrak{Sch}}
\newcommand{\Var}{\mathfrak{Var}}
\newcommand{\Rings}{\mathfrak{Rings}}
\DeclareMathOperator{\In}{in}
\DeclareMathOperator{\Ext}{Ext}
\DeclareMathOperator{\Spe}{Sp\acute{e}}
\DeclareMathOperator{\HHom}{\mathscr{H}\!\mathit{om}}
\newcommand{\isoto}{\overset{\sim}{\to}}
\newcommand{\isolongto}{\overset{\sim}{\longrightarrow}}
\newcommand{\Mod}{\mathsf{mod}\mathchar`-}
\newcommand{\MOD}{\mathsf{Mod}\mathchar`-}
\newcommand{\gr}{\mathsf{gr}\mathchar`-}
\newcommand{\qgr}{\mathsf{qgr}\mathchar`-}
\newcommand{\uqgr}{\underline{\mathsf{qgr}}\mathchar`-}
\newcommand{\qcoh}{\mathsf{qcoh}\mathchar`-}
\newcommand{\Alg}{\mathsf{Alg}\mathchar`-}
\newcommand{\coh}{\mathsf{coh}\mathchar`-}
\newcommand{\vect}{\mathsf{vect}\mathchar`-}
\newcommand{\imm}[1][imm]{\hspace{0.75ex}\raisebox{0.58ex}{%
\begin{tikzpicture}[commutative diagrams/every diagram]
\draw[commutative diagrams/.cd, every arrow, every label,hook,{#1}] (0,0ex) -- (2.25ex,0ex);
\end{tikzpicture}}\hspace{0.75ex}}
\newcommand{\dashto}[2]{\smash{\hspace{-0.7em}\begin{tikzcd}[column sep=small,ampersand replacement=\&] {#1} \rar[dashed] \& {#2} \end{tikzcd}\hspace{-0.7em}}}

\usepackage{todonotes}

\title{Hartshorne Ch.~I, \S2 Projective Varieties}
\author{Takumi Murayama, Kyu Jun}

\begin{document}
\maketitle
\setcounter{section}{2}
\begin{problem}\label{exc:2.1}
  Prove the ``homogeneous Nullstellensatz,'' which says if $\mathfrak{a}
  \subseteq S$ is a homogeneous ideal, and if $f \in S$ is a homogeneous
  polynomial with $\deg f > 0$, such that $f(P) = 0$ for all $P \in
  Z(\mathfrak{a})$ in $\PP^n$, then $f^q \in \mathfrak{a}$ for some $q >
  0$.
\end{problem}
\begin{proof}
  Consider $S = k[x_0,\ldots,x_n]$ as an affine polynomial ring, and let
  $CZ(\mathfrak{a})$ be the affine variety defined by $\mathfrak{a}$ in
  $\AA^{n+1}$.
  Then, $f(P) = 0$ for any $P \in CZ(\mathfrak{a})$, since any nonzero $P$ is a
  representative of some point in $Z(\mathfrak{a})$, and if $P=0$ then $f(P) =
  0$ by the fact that $f$ is homogeneous with $\deg f > 0$.
  By the affine Nullstellensatz (Thm.~1.3A), we have $f^q \in \mathfrak{a}$ for
  some $q > 0$.
\end{proof}

\begin{problem}\label{exc:2.2}
  For a homogeneous ideal $\mathfrak{a} \subseteq S$, show that the following
  conditions are equivalent:
  \begin{enumi}
    \item $Z(\mathfrak{a}) = \emptyset$ (the empty set);
    \item $\sqrt{\mathfrak{a}} =$ either $S$ or the ideal $S_+ =
      \bigoplus_{d > 0}S_d$;
    \item $\mathfrak{a} \supseteq S_d$ for some $d > 0$. 
  \end{enumi}
\end{problem}
\begin{proof}
  $(i) \Rightarrow (ii)$. If $Z(\mathfrak{a}) = \emptyset$, then
  $CZ(\mathfrak{a}) = \{0\}$ or $\emptyset$, where as in Exercise \ref{exc:2.1}
  $CZ(\mathfrak{a})$ is the affine variety in $\AA^{n+1}$ defined by
  $\mathfrak{a}$.
  If $CZ(\mathfrak{a}) = \{0\}$, then $x_i^r \in \mathfrak{a}$ by the affine
  Nullstellensatz (Thm.~1.3A) for each $0 \le i \le n$, hence $x_i \in
  \sqrt{\mathfrak{a}}$ for each $0 \le i \le n$, and so $\sqrt{\mathfrak{a}}
  \supset S_+$; this is moreover an equality since $S_0 = k$, and $\mathfrak{a}
  \cap k = \emptyset$ for otherwise $CZ(\mathfrak{a}) = \emptyset$.
  On the other hand, if $CZ(\mathfrak{a}) = \emptyset$, then since $\emptyset =
  Z(S)$ by the proof of Prop.~1.1, then $I(\emptyset) = I(CZ(\mathfrak{a})) =
  \sqrt{\mathfrak{a}} = S$ by Prop.~$1.2(d)$.
  \par $(ii) \Rightarrow (iii)$. If $\sqrt{\mathfrak{a}} = S$, then $1 \in
  \sqrt{\mathfrak{a}}$, which implies $1 = 1^n \in \mathfrak{a}$, and so $\mathfrak{a} \supset S_d$ for any $d$.
  On the other hand, if $\sqrt{\mathfrak{a}} = S_+$, then $x_i^r \in
  \mathfrak{a}$ for some $r$ for all $0 \le i \le n$. But every element in
  $S_{r(n+1)}$ has some $x_i^r$ appearing in each term by the pigeon-hole
  principle, and so $\mathfrak{a} \supset S_{r(n+1)}$.
  \par $(iii) \Rightarrow (i)$. $\mathfrak{a} \supset S_d$ implies all $x_i^d$
  must vanish on $Z(\mathfrak{a})$, which is impossible in $\PP^n$, hence
  $Z(\mathfrak{a}) = \emptyset$.
\end{proof}

\begin{problem}\mbox{}\label{exc:2.3}
  \begin{enuma}
    \item If $T_1 \subseteq T_2$ are subsets of $S^h$, then $Z(T_1) \supseteq
      Z(T_2)$.
    \item If $Y_1 \subseteq Y_2$ are subsets of $\PP^n$, then $I(Y_1) \supseteq
      I(Y_2)$.
    \item For any two subsets $Y_1,Y_2$ of $\PP^n$, $I(Y_1 \cup Y_2) = I(Y_1)
      \cap I(Y_2)$.
    \item If $\mathfrak{a} \subseteq S$ is a homogeneous ideal with
      $Z(\mathfrak{a}) \ne \emptyset$, then $I(Z(\mathfrak{a})) =
      \sqrt{\mathfrak{a}}$.
    \item For any subset $Y \subseteq \PP^n$, $Z(I(Y)) = \overline{Y}$.
  \end{enuma}
\end{problem}
\begin{proof}[Proof of $(a)$]
  $P \in Z(T_2)$ implies $f(P) = 0$ for all $f \in T_2$, and so in particular
  $f(P)= 0$ for all $f \in T_1$, hence $P \in Z(T_1)$.
\end{proof}
\begin{proof}[Proof of $(b)$]
  It suffices to show the homogeneous elements of $I(Y_2)$ are in $I(Y_1)$,
  since both ideals are assumed to be homogeneous. If
  $f$ is a generator of $I(Y_2)$, then $f(P) = 0$ for all $P \in Y_2$, and in
  particular $f(P) = 0$ for all $P \in Y_1$, hence $f \in I(Y_1)$.
\end{proof}
\begin{proof}[Proof of $(c)$]
  As in $(b)$, it suffices to show the homogeneous elements of $I(Y_1 \cup
  Y_2)$ are in $I(Y_1) \cap I(Y_2)$, and vice versa. But $f \in I(Y_1 \cup Y_2)$
  if and only if $f(P) = 0$ for all $P \in Y_1 \cup Y_2$, which holds if and
  only if $f(P) = 0$ for all $P \in Y_1$ and $P \in Y_2$. But this is equivalent
  to $f \in I(Y_1) \cap I(Y_2)$ by definition.
\end{proof}
\begin{proof}[Proof of $(d)$]
  $\sqrt{\mathfrak{a}} \subset I(Z(\mathfrak{a}))$, since any power of a
  homogeneous element $f \in \mathfrak{a}$
  vanishes at all points of $Z(\mathfrak{a})$ by definition.
  \par Conversely, suppose $f \in I(Z(\mathfrak{a}))$. Then, $f(P) = 0$ for all
  $P\in Z(\mathfrak{a})$, hence by the homogeneous Nullstellensatz (Exercise
  \ref{exc:2.1}), $f^q \in \mathfrak{a}$ for some $q>0$, i.e., $f \in
  \sqrt{\mathfrak{a}}$.
\end{proof}
\begin{proof}[Proof of $(e)$]
  Because $Z(I(Y))$ is closed, by the definition of the closure, $Z(I(Y))
  \supset Y$ implies $Z(I(Y)) \supset \overline{Y}$. Now let $W$ be any closed
  set containing $Y$. Then $W = Z(\mathfrak{a})$ for some homogeneous ideal
  $\mathfrak{a}$. Then from (b), (d) and the fact that an ideal is contained in
  its radical ideal, $\mathfrak{a} \subset \sqrt{\mathfrak{a}}=
  I(Z(\mathfrak{a})) \subset I(Y)$. Hence, $\mathfrak{a} \subset I(Y)$. So from
  $(a)$, we have $W = Z(\mathfrak{a}) \supset Z(I(Y))$. Hence, $Z(I(Y)) =
  \overline{Y}$.
\end{proof}

\begin{problem}\mbox{}\label{exc:2.4}
  \begin{enuma}
    \item There is a one-to-one inclusion-reversing correspondence between
      algebraic sets in $\PP^n$ and homogeneous radical ideals of $S$ not equal
      to $S_+$ given by $Y \mapsto I(Y)$ and $\mathfrak{a} \mapsto
      Z(\mathfrak{a})$.
    \item An algebraic set $Y \subseteq \PP^n$ is irreducible if and only if $I(Y)$ is a prime ideal. 
    \item Show that $\PP^n$ itself is irreducible. 
   \end{enuma}
\end{problem}
\begin{proof}[Proof of $(a)$]
  The correspondence is one-to-one by Exercise $\ref{exc:2.3}(d),(e)$, where we
  note that the correspondence is with homogeneous radical ideals of $S$ not
  equal to $S_+$ by Exercise \ref{exc:2.2}, and is inclusion-reversing by
  Exercise $\ref{exc:2.3}(a),(b)$.
\end{proof}
\begin{proof}[Proof of $(b)$]
  Suppose $Y$ is irreducible. Recall from p.~9 that to check primeness of
  $I(Y)$, it suffices to check that for any two homogeneous $f,g \in I(Y)$, $fg
  \in I(Y)$ implies either $f \in I(Y)$ or $g \in I(Y)$. So let $fg \in I(Y)$;
  then, $Y \subset Z(fg) = Z(f) \cup Z(g)$. Thus, $Y = (Y \cap Z(f)) \cup (Y \cap
  Z(g))$, which are both closed subsets of $Y$. Since $Y$ is irreducible, either
  $Y = Y \cap Z(f)$ or $Y = Y \cap Z(g)$, in which case $Y \subset Z(f)$ or
  $Y \subset Z(g)$, respectively. Thus either $f \in I(Y)$ or $g \in I(Y)$,
  respectively.
  \par Conversely, suppose $\mathfrak{p}$ is a homogeneous prime ideal, and
  suppose that $Z(\mathfrak{p}) = Y_1 \cup Y_2$. Then, $\mathfrak{p} = I(Y_1)
  \cap I(Y_2)$ by Exercise $\ref{exc:2.3}(c)$, so either $\mathfrak{p} = I(Y_1)$
  or $\mathfrak{p} = I(Y_2)$. Thus, $Z(\mathfrak{p}) = Y_1$ or $Y_2$,
  respectively, hence it is irreducible.
  %Let $Y= Z(T)$ where $T \subset k[x_0, \cdots, x_n]$ is a set of homogeneous
  %polynomials. Let $CY \subset \AA^{n+1}$ be the affine variety defined by $T$.
  %Then, since the projection map $\AA^{n+1} \to
  %(\AA^{n+1}-\{(0,\cdots,0)\})/k^{\times} = \PP^{n}$ is surjective and
  %continuous, $Y$ is irreducible if and only if $CY$ is irreducible. From
  %Cor.~$1.4$, $CY$ is irreducible if and only if $I(CY)$ is prime. But $I(CY)$
  %is prime if and only if $I(Y)$ is prime. Hence, $Y$ is irreducible if and only
  %if $I(Y)$ is prime. 
\end{proof}
\begin{proof}[Proof of $(c)$]
  $\PP^n = Z(0)$, and $(0) \subset k[x_0,\ldots,x_n]$ is a homogeneous prime
  ideal, hence $\PP^n$ is irreducible by $(b)$.
  %There are several ways to show $\PP^n$ is irreducible. One way you can do is to use that $U_0 \simeq \AA^n$ is irreducible, and $\PP^n = \overline{U_0}$ is irreducible as it is a closure of an irreducible subset (example 1.14). 
  %One another way is to see that $\PP^n = Z(\mathfrak{o})$, where $\mathfrak{o} = (0)$. As $(0)$ is a homogeneous prime ideal, $\PP^n$ is irreducible (from (b).  
\end{proof}

\begin{problem} \mbox{}
  \begin{enuma}
    \item $\PP^n$ is a noetherian topological space. 
    \item Every algebraic set in $\PP^n$ can be written uniquely as a finite
      union of irreducible algebraic sets, no one containing another. These are
      called its \emph{irreducible components}. 
  \end{enuma}
\end{problem}
\begin{proof}[Proof of $(a)$]
  Let $Y_1 \supset Y_2 \supset Y_3 \supset \cdots$ be a descending chain of
  closed subsets of $\PP^n$; then, $I(Y_1) \subset I(Y_2) \subset \cdots$ is an
  ascending chain of homogeneous radical ideals in $S = k[x_0, \cdots, x_n]$
  (from Exercise $\ref{exc:2.4}(a)$). Then, this chain of ideals eventually
  stabilizes since $S$ is a noetherian ring. For each $i$, $Y_i = Z(I(Y_i))$, and
  so the descending chain of $Y_i$ also eventually stabilizes.
\end{proof}
\begin{proof}[Proof of $(b)$]
  From $(a)$, $\PP^n$ is noetherian, so by Prop.~$1.5$, every algebraic set in
  $\PP^n$ can be written uniquely as a finite union of irreducible closed
  subsets of $\PP^n$, which are algebraic by definition, such that no one
  contains another.
\end{proof}

\begin{problem}\label{exc:2.6}
  If $Y$ is a projective variety with homogeneous coordinate ring $S(Y)$, show that $\dim S(Y) = \dim Y + 1$. 
\end{problem}
%\begin{lemma}\label{lem:2.6}
%  Let $\varphi_i\colon U_i \to \AA^n$ be the homeomorphism of Prop.~$2.2$, and
%  let $Y_i = \varphi(Y \cap U_i)$. If $Y_i \ne \emptyset$, then $\dim Y_i = \dim
%  Y$.
%\end{lemma}
%\begin{proof}[Proof of Lemma $\ref{lem:2.6}$]
%  Note first $U_i$ is dense in $\PP^N$ hence $Y_i$ is dense in $Y$.
%  Now suppose $Z_0 \subset Z_1 \subset \cdots \subset Z_n$ is a sequence of
%  distinct closed irreducible subsets of $Y_i$. Then, $\overline{Z_0} \subset
%  \overline{Z_1} \subset \cdots \subset \overline{Z_n}$ is a sequence of
%  distinct closed irreducible subsets of $\overline{Y}$ by Ex.~$1.1.4$, hence
%  $\dim Y_i \le \dim Y$. In particular, $\dim Y_i < \infty$ since $\dim Y$ is
%  bounded by the dimension of the projective space $\PP^N$ containing $Y$.
%  
%  so choose a
%  maximal chain of irreducible closed subsets $Z_0 \subset \cdots \subset Z_n$,
%  with $n = \dim Y$. We can choose such a chain where $Z_0 \subset Y_i$ since 
%\end{proof}
\begin{proof}
  Let $\varphi_i\colon U_i \to \AA^n$ be the homeomorphism of Prop.~$2.2$, and
  let $Y_i = \varphi(Y \cap U_i)$.
  Let $A(Y_i)$ is the affine coordinate ring of $Y_i$.
  We want to show that $A(Y_i)$ can be identified with the subring
  $(S(Y)_{x_i})_0$ of elements of degree $0$ of the localized ring $S(Y)_{x_i}$.
  If $x_i \in I(Y)$, then $S(Y)_{x_i} = A(Y_i) = 0$, hence the identification is
  trivial. Otherwise, any element $\frac{f}{x_i^e} \in (S(Y)_{x_i})_0$ can be
  expressed as
  \[
    f\left(\frac{x_0}{x_i}, \cdots, \frac{x_{i-1}}{x_i}, 1, \frac{x_{i+1}}{x_i},
    \cdots, \frac{x_n}{x_i}\right),
  \]
  which is $\alpha(f) \in A(Y_i)$, where $\alpha$ is the function defined in Prop
  $2.2$. Conversely, for $f \in A(Y_i)$ of degree $e$, consider the function
  $\beta\colon A(Y_i) \to (S(Y)_{x_i})_0$ where $f \mapsto x_i^{-e}f$. Then
  $\alpha \circ \beta$ and $\beta \circ \alpha$ are identity maps. Thus,
  $A(Y_i)$ can be identified with $(S(Y)_{x_i})_0$, and so $S(Y)_{x_i} =
  ((S(Y)_{x_i})_0)[x_i, x_i^{-1}] \cong A(Y_i)[x_i, x_i^{-1}]$.
  \par From Thm.~$1.8$A, the transcendence degree of $K(A(Y_i)[x_i, x_i^{-1}]) =
  K(A(Y_i))(x_i)$ is one plus the transcendence degree of $K(A(Y_i))$. From
  Prop.~$1.7$, the transcendence degree of $K(A(Y_i)) = \dim Y_i$. Hence,  
  \[\dim S(Y) = \dim S(Y)_{x_i} = \dim Y_i + 1 \le \dim Y+1,\]
  where $\dim Y_i \le \dim Y$ by Exercise $1.10(a)$. Conversely, $\dim Y + 1 \le
  \dim S(Y)$ since an ascending sequence of irreducible closed subsets of $Y$
  corresponds by Exercise $\ref{exc:2.4}$ to an descending sequence of
  homogeneous prime ideals not equal to $S_+$, and such a sequence can be
  extended to one of length one higher by starting with $S_+$.
  \par Note we have shown that $\dim Y_i = \dim Y$ as long as $Y_i \ne
  \emptyset$.
\end{proof}

\begin{problem}\mbox{}\label{exc:2.7}
  \begin{enuma}
    \item $\dim \PP^n = n$.
    \item If $Y \subseteq \PP^n$ is a quasi-projective variety, then $\dim Y =
      \dim \overline{Y}$.
  \end{enuma}
\end{problem}
\begin{proof}[Proof of $(a)$]
  From Exercise $\ref{exc:2.6}$,
  \begin{equation*}
    n + 1 = \dim k[x_0,\ldots,x_n] = \dim S(\PP^n) = \dim(\PP^n) +1,
  \end{equation*}
  hence $\dim \PP^n = n$.
\end{proof}
\begin{proof}[Proof of $(b)$]
  Let $Y_i \coloneqq Y \cap U_i$ be nonempty. Then, by our remark at the end of
  the proof of Exercise $\ref{exc:2.6}$, $\dim Y_i = \dim Y$. Similarly, $\dim
  \overline{Y} = \dim Z_i$, where $Z_i$ is the closure of $Y_i$ in $U_i$. Thus,
  it suffices to show $\dim Z_i = \dim Y_i$, but this is exactly Prop.~$1.10$.
  %From Exercise $\ref{exc:2.6}$, $\dim S(Y) = \dim Y +1$. We note that $\dim S(Y) = \dim (CY)$, where $CY = \pi^{-1} (Y)$. $\pi : \AA^{n+1} \setminus \{0\} \to \PP^n$. This is true because $S(Y) = k[x_0, \cdots, x_n]/I(Y)$, and realizing this in an affine space $\AA^{n+1}$, we get $\dim S(Y) = \dim CY$(from prop 1.7). Then, from prop 1.10, $\dim S(Y) = \dim CY = \dim \overline{CY} = \dim C \overline{Y} = \dim S(\overline{Y})$. Hence, $\dim Y = \dim S(Y) -1 = \dim S(\overline{Y}) -1 = \dim \overline{Y}$.
\end{proof}

\begin{problem} A projective variety $Y \subseteq \PP^n$ has dimension $n-1$ if
  and only if it is the zero set of a single irreducible homogeneous polynomial
  $f$ of positive degree. $Y$ is called a \emph{hypersurface} in $\PP^n$. 
\end{problem}
\begin{proof}
  Let $f$ be an irreducible homogeneous polynomial of positive degree. Then,
  $Y \coloneqq Z(f)$ has coordinate ring $k[x_0,\ldots,x_n]/(f)$, where $(f)$ is
  prime since $f$ is irreducible. By the Hauptidealsatz (Thm.~$1.11$A), $(f)$ has
  height $1$, so by Thm.~$1.8$A$(b)$, $\dim S(Y) = n$, and $\dim Y = n-1$ by
  Exercise $\ref{exc:2.6}$.
  \par Conversely, suppose $Y$ has dimension $n-1$. Then,
  \[ \dim S(Y) = \dim k[x_0,\ldots,x_n]/I(Y) = n\]
  by Exercise $\ref{exc:2.6}$, where $I(Y)$ is
  prime since $Y$ is irreducible by Exercise $\ref{exc:2.4}(b)$. Now $I(Y)$ is a
  height one prime in $k[x_0,\ldots,x_n]$ by Thm.~$1.8$A$(b)$, hence is
  generated by a single nonconstant irreducible polynomial of positive degree
  by Prop.~$1.12$A; it can moreover be taken to be homogeneous since $I(Y)$ is a
  homogeneous ideal, hence is generated by homogeneous elements.
  %Let $Y$ be a projective variety of dimension $n-1$. Then, $\dim S(Y) = n$ from
  %Exercise $\ref{exc:2.6}$. Regarded as a polynomial in $k[x_0,\cdots,x_n]$
  %(i.e. considering $Z( S(Y))$ an affine space $\AA^{n+1}$), $CY = Z(S(Y))$ is
  %a $n$-dimensional variety. Using prop 1.13, an affine cone $CY$ of $Y$ is the
  %zero set $Z(f)$ of a single nonconstant irreducible polynomial. Thus, $Y =
  %Z(f')$ where $f'$ the homogenization of $f$. % This just isn't true...
  %\par Conversely, let $Y= Z(F)$ where $f \in  S = k[x_0, \cdots, x_n]$ be an irreducible homogeneous polynomial of positive degree (thus $f \neq 0$). By Hauptidealsatz (theorem 1.11A), $(f)$ has height 1, as $(f)$ is a minimal prime ideal containing $f$. Hence, in an affine cone, $CY = Z(f)$ has dimension $n+1-1 =n$ (by theorem 1.8A). Hence, $\dim Y = \dim CY-1 = n-1$. 
  %\par cf. It is quite obvious (to Takumi who is quite unhappy to see the phrase "quite obvious": I know you are unhappy, but I-perhaps you- will show this soon! :p) that $\dim Y = \dim CY -1$, and we will show that in exercise 2.10 (c).
  %I dunno Kyu, this problem is super simple without talking about affine cones
  %at all...
\end{proof}

\begin{problem}[Projective Closure of an Affine Variety]
  If $Y \subseteq \AA^n$ is an affine variety, we identify $\AA^n$ with an open
  set $U_0 \subset \PP^n$ by the homeomorphism $\varphi_0$. Then we can speak of
  $\overline{Y}$, the closure of $Y$ in $\PP^n$, which is called the
  \emph{projective closure} of $Y$. 
 \begin{enuma}
   \item Show that $I(\overline{Y})$ is the ideal generated by $\beta(I(Y))$,
     using the notation of the proof of $(2.2)$. 
   \item Let $Y \subset \AA^n$ be the twisted cubic of \emph{(Ex $1.2$)}. Its
     projective closure $\overline{Y} \subset \PP^n$ is called the \emph{twisted
     cubic curve} in $\PP^3$. Find generators for $I(Y)$ and $I(\overline{Y})$,
     and use this example to show that if $f_1, \ldots, f_r$ generate $I(Y)$,
     then $\beta(f_1), \ldots, \beta(f_r)$ do \emph{not} necessarily generate
     $I(\overline{Y})$. 
 \end{enuma}
\end{problem}
\begin{proof}[Proof of $(a)$]
  Let $J$ be the ideal generated by $\beta(I(Y))$. Since $\varphi_0^{-1}(Y) =
  Z(\beta(I(Y)) \cap U_0 \subset Z(\beta(I(Y)))$, we have
  $\overline{Y} \subset Z(\beta(I(Y)))$ by the
  definition of closure. Thus, $I(\overline{Y}) \supset I(Z(\beta(I(Y))))$
  by Exercise $\ref{exc:2.3}(b)$. Finally, $I(Z(\beta(I(Y)))) = I(Z(J)) = \sqrt{J}
  \supset J$ by Exercise $\ref{exc:2.3}(d)$, hence $I(\overline{Y}) \supset J$.
  \par Conversely, we have $\varphi_0^{-1}(Y) \subset \overline{Y}$, hence
  $I(\varphi_0^{-1}(Y)) \supset I(\overline{Y})$ by Exercise $\ref{exc:2.3}(b)$. But
  $J \supset I(\varphi_0^{-1}(Y))$ since any element $F \in
  I(\varphi_0^{-1}(Y))$ has $\alpha(F) \in I(Y)$, hence $F = \beta(\alpha(F))
  \in \beta(I(Y))$. Thus, $J \supset I(\overline{Y})$ and we conclude that $J =
  I(\overline{Y})$.
\end{proof}
\begin{proof}[Proof of $(b)$]
  From Exercise $1.2$, $I(Y) = (y-x^2, z-x^3)$.
  We claim $I(\overline{Y}) = (wy-x^2,xz-y^2,wz-xy) \subset k[w,x,y,z]$.
  $\supset$ is clear since
  \begin{alignat*}{4}
    wy &{}- x^2 &{}={}& \beta(y-x^2)\\
    xz &{}- y^2 &{}={}& \beta(xz-y^2) &{}={}& \beta(x(z-x^3)-(y+x^2)(y-x^2))\\
    wz &{}- xy  &{}={}& \beta(z-xy)   &{}={}& \beta((z-x^3) - x(y-x^2))
  \end{alignat*}
  and then since $I(\overline{Y}) \supset \beta(I(Y))$ by $(a)$.
  \par Conversely, we want to show $I(\overline{Y}) \subset
  (wy-x^2,xz-y^2,wz-xy)$. First, we have
  \begin{equation*}
    Z(wy-x^2,xz-y^2,wz-xy) \cap U_0 \subset \varphi_0^{-1}(Y),
  \end{equation*}
  for if $[w:x:y:z] \in Z(wy-x^2,xz-y^2,wz-xy) \cap U_0$, then we can let $w=1$,
  so $[1:x:y:z] \in Z(wy-x^2,xz-y^2,wz-xy)$ satisfies $y-x^2$, $xz - y^2$, and
  $z - xy$, hence also satisfies $z - x^3$ by combining the first and third
  relations, and therefore $[1:x:y:z] \in 
  \varphi_0^{-1}(Y)$. Taking closures, we then have $Z(wy-x^2,xz-y^2,wz-xy)
  \subset \overline{Y}$.
  \par To show $I(\overline{Y}) \subset
  (wy-x^2,xz-y^2,wz-xy)$, then, by Exercise $\ref{exc:2.3}(d)$ it suffices to
  show $(wy-x^2,xz-y^2,wz-xy)$ is radical. We will moreover show
  $(wy-x^2,xz-y^2,wz-xy)$ is prime; by Exercise $\ref{exc:2.4}(b)$, it suffices
  to show $Z(wy-x^2,xz-y^2,wz-xy)$ is irreducible. We note that the image of an
  irreducible variety through a regular map $f\colon X \to X'$ is irreducible, for
  otherwise we would get a decomposition of $X$ by taking the preimage of the
  decomposition of $f(X')$. So, we claim $Z(wy-x^2,xz-y^2,wz-xy)$ is the image
  of the regular map
  \begin{align*}
    f\colon \mathbb{P}^1 &\to X_2\\
    [s:t] &\mapsto [s^3:s^2t:st^2:t^3]
  \end{align*}
  We note that the image is contained in $Z(wy-x^2,xz-y^2,wz-xy)$ since the
  coordinates satisfy the relations
  \begin{gather*}
    wy=(s^3)(st^2) = s^4t^2 = (s^2t)^2 = x^2,\\
    xz=(s^2t)(t^3) = s^2t^4 = (st^2)^2 = y^2,\\
    wz=(s^3)(t^3)=s^3t^3=(s^2t)(st^2)=xy.
  \end{gather*}
  It remains to show any point $[w:x:y:z] \in Z(wy-x^2,xz-y^2,wz-xy)$ has
  preimage in $\mathbb{P}^1$. In $Z(wy-x^2,xz-y^2,wz-xy)$, either $w \ne 0$ or $z \ne 0$, for otherwise $x=y=0$ as well using $wy=x^2$ and $xz=y^2$. If $w \ne 0$, then consider the point $(w^{1/3},xw^{-2/3}) \in k^2$, for arbitrary choice of cube root (which exists since $k$ is algebraically closed). Then,
  \begin{equation*}
    f\left[w^{1/3}:\frac{x}{w^{2/3}}\right] = \left[ w:w^{2/3} \cdot
    \frac{x}{w^{2/3}}:w^{1/3} \cdot \frac{x^2}{w^{4/3}}:
    \frac{x^3}{w^{2}}\right] = \left[ w:x:\frac{x^2}{w}:
    \frac{x^3}{w^{2}}\right],
  \end{equation*}
  using $wy=x^2$, $x^2w^{-1} = y$. Using the same equation, $x^3w^{-2} = wxyw^{-2} = xyw^{-1}$, and using $wz=xy$, $xyw^{-1} = z$. Thus, $[w^{1/3}:xw^{-2/3}] \mapsto [w:x:y:z]$. Now if $z \ne 0$, then consider the point $(yz^{-2/3},z^{1/3}) \in k^2$, for arbitrary choice of cube root. Then,
  \begin{equation*}
    f\left[ \frac{y}{z^{2/3}}:z^{1/3} \right] = \left[ \frac{y^3}{z^2}:\frac{y^2}{z^{4/3}} \cdot z^{1/3}:\frac{y}{z^{2/3}} \cdot z^{2/3}:z \right] = \left[ \frac{y^3}{z^2}:\frac{y^2}{z}:y:z \right],
  \end{equation*}
  using $xz = y^2$, $y^2z^{-1} = x$. Using the same equation, $y^3z^{-2} =
  xyzz^{-2} = xyz^{-1}$, and using $wz = xy$, $xyz^{-1} = w$. Thus,
  $[yz^{-2/3}:z^{1/3}] \mapsto [w:x:y:z]$. This shows $f(\PP^1) =
  Z(wy-x^2,xz-y^2,wz-xy)$, hence $Z(wy-x^2,xz-y^2,wz-xy)$ is irreducible, and by
  the above $I(\overline{Y}) = (wy-x^2,xz-y^2,wz-xy)$.
  \par Finally, we note that letting $f_1 = y-x^2,f_2 = z-x^3$, we have
  \[I(\overline{Y}) \setminus (\beta(f_1),\beta(f_2)) =
    I(\overline{Y}) \setminus (wy-x^2,w^2z-x^3) \ni wz-xy,\]
  since every homogeneous degree two polynomial in $I(\overline{Y})$ is a
  $k$-multiple of $wy-x^2$.
\end{proof}

\begin{problem} \textbf{The Cone Over a Projective Variety} Let $Y \subset \PP^n$ be a nonempty algebraic set, and let $\theta : \AA^{n+1} \setminus \{(0,\cdots,0\} \to \PP^n$ be the map which sends the point with affine coordinates $(a_0, \cdots, a_n)$ to the point with homogeneous coordinates $[a_0 :\cdots: a_n]$. We define the affine cone over $Y$ to be $$C(Y) = \theta^{-1}(Y) \cup \{(0,\cdots, 0)\}.$$ 
  \begin{enuma}
    \item Show that $C(Y)$ is an algebraic set in $\AA^{n+1}$, whose ideal is equal to $I(Y)$, considered as an ordinary ideal in $k[x_0, \cdots, x_n]$. 
    \item $C(Y)$ is irreducible if and only if $Y$ is irreducible. 
    \item $\dim C(Y) = \dim Y +1$
  \end{enuma}
\end{problem}
\begin{proof}[proof of (a)]
$I(C(Y)) = I(\theta^{-1}(Y) \cup {(0, \cdots, 0)}) = I(Y)$ because if $f \in I(Y)$, $f(cy) =0$ for any constant $c \in k$. Hence, $f \in I(C(Y)$, because $C(Y)$ consists of all scalar multiples of elements of Y. Hence, $I(Y) \subseteq I(C(Y))$. Obviously, $I(Y) \supseteq I(C(Y))$ as we can embed $Y$ into $C(Y)$. Hence, $C(Y) = Z(I(Y))$ is algebraic, and its ideal is equal to $I(Y)$, considered as an ordinary ideal in $k[x_0, \cdots, x_n]$.
\end{proof}
\begin{proof} [proof of (b)]
From corollary 1.4, an algebraic set is irreducible iff its ideal is a prime ideal. Hence, $C(Y) \mbox{ irreducible} \iff I(C(Y)) = I(Y) \mbox{ is prome} \iff Y$ is irreducible. 
\end{proof}
\begin{proof} [proof of (c)]
$\dim C(Y) = \dim S(y) = \dim Y +1$ by problem 2.6. We can also show that $\dim
C(Y) = \dim Y +1$ by directly using the definition of the dimension (pg 5). If
$\dim Y = n$, there is a chain $Z_0 \subsetneq Z_1 \subsetneq \cdots\subsetneq
Z_n \subseteq Y$ of irreducible variates. An origin is added in $C(Y)$, and we can add the point (which is irreducible) to the beginning of the given chain. Hence, $\dim C(Y) = n+1$. 
\end{proof}


\begin{problem} \textbf{Linear Varieties in $\PP^N$} A hypersurface defined by a linear polynomial is called a hyperplane. 
\begin{enuma}
\item Show that the following two conditions are equivalent for a variety $Y \subset \PP^n$: 
\begin{enumi}
\item $I(Y)$ can be generated by linear polynomials
\item $Y$ can be written as an intersection of hyperplanes. 
\end{enumi}
\item If $Y$ is a linear variety of dimension $r$ in $\PP^n$, show that $I(Y)$ is minimally generated by $n-r$ linear polynomials 
\item Let $Y,Z$ be linear varieties in $\PP^n$, with $\dim Y = r$, $\dim Z = s$. If $r+s-n \geq 0$, then $Y \cap Z \neq \emptyset$. Furthermore, if $Y \cap Z \neq \emptyset$, then $Y \cap Z$ is a linear variety of dimension $\geq r+s-n$ (Think of $\AA^{n+1}$ as a vector space over $k$, and work with its subspaces.)
\end{enuma}
\end{problem}

\begin{proof}[proof of (a)] \mbox{}
\begin{enumi}
\item $\Rightarrow$ (ii): Assume that $I(Y) = (f_1, \cdots, f_m)$ where $f_i$ are linear polynomials. Then, $Y = Z(I(Y)) = Z(f_1, \cdots, f_m) = \cap_{i} Z(f_i)$. Thus, $Y$ can be written as an intersection of hyperplanes. 

\item $\Rightarrow$ (1): Let $Y = Z(f_1) \cap \cdots \cap Z(f_m)$, where $f_i$'s are linear polynomials. Then, $Y = Z(f_1) \cap \cdots \cap Z(f_m) = Z(f_1, \cdots, f_m) \Rightarrow I(Y) = rad(f_1, \cdots, f_m) = (f_1, \cdots, f_m)$ because $f_i$'s are linear.
\end{enumi}
\end{proof}
\begin{proof}[proof of (b)] Proof by contradiction: Assume that $Y$ is a linear variety of dimension $r$, but $I(Y)$ is minimally generated by at least $m >n-r$ linear polynomials. Then, we see that $I(Y) = (f_1, \cdots, f_m)$ where each $f_i$'s are linear, and $\{f_i\}$ is linearly independent. Then, $\dim S(Y) = \dim (S/I(Y)) = n+1-m < n-n+r+1 = r+1$. Hence, from exercise 2.6, we conclude that $\dim Y = \dim S(Y) -1 < r+1-1 = r \Rightarrow \dim Y < r$, which is a contradiction. Hence, if $Y$ is a linear variety of dimension $r$ in $\PP^n$, $I(Y)$ is a linear variety of dimension $r$, $Y$ is minimally generated by at most $n-r$ linear polynomials. On the other hand, if it is generated by $m <n-r$ elements, $\dim S(Y) = \dim (S/I(Y)) > n-n+r+1 = r+1\Rightarrow \dim Y = \dim S(Y) -1 > r$. Hence, $I(Y)$ is minimally generated by $n-r$ linear polynomials.
\end{proof}

\begin{proof}[proof of (c)] Let $Y,Z$ be linear varieties in $\PP^n$, with $\dim Y = r$, $\dim Z = s$ where $r+s-n \geq 0$. We need to show that $Y \cap Z \neq \emptyset$. Now as $\dim Y = r$, $\dim Z = s$, respectively, from (b) we conclude that $I(Y)$ (resp. $I(Z)$) is minimally generated by $n-r$ ( (resp. $n-s$) linear polynomials $\{f_1, \cdots , f_{n-r}\}$ (resp. $\{g_1, \cdots g_{n-s}$\}). i.e. $I(Y) = (f_1, \cdots, f_{n-r})$, $I(Z) = (g_1, \cdots g_{n-s})$. Then, $I(Y), I(Z) \subset (f_1, \cdots, f_{n-r}, g_1, \cdots, g_{n-s}) = J$. As $r+s-n \geq 0$, $J$ is minimally generated by at most $2n-s-r \leq n$ linear polynomials. As $I(Y), I(Z) \subset J \Rightarrow Y, Z \supset Z(J) \Rightarrow Y \cap Z \supset Z(J)$, we only need to show that $Z(J) \neq \emptyset$ to show that $Y \cap Z \neq \emptyset$. By weak Nullstellensatz, (i.e. AM exercise 17 pg 69) $Z(J) \neq \emptyset$ if $rad(J) = J \neq (1)$. This is true because $J$ is minimally generated by at most $2n-s-r \leq n$ linear polynomials. (in order for $J = (1)$, we need at least $n+1$ linearly independent linear polynomials). Hence, $Y \cap Z \neq \emptyset$

\par Now, assume that $Y \cap Z \neq \emptyset$. Then, from (a), $Y = \cap_i Z(f_i)$ and $Z = \cap_j Z(g_j)$ for for $f_i, g_j$ : linear polynomials. From (b), we can choose $f_i$, $g_j$ such that $i = 1, \cdots, n-r$, and $j = 1, \cdots n-s$. Then, $Y \cap Z = (\cap_i Z(f_i)) \cap (\cap_j Z(g_j))$, which is again an intersection of hyperplanes. Thus, $I(Y\cap Z)$ can be generated by $(r+s)$ linear polynomials $f_i, g_j$. That is, $Y \cap Z$ is a linear variety such that $I(Y\cap Z)$ is minimally generated by at most $2n-r-s$ linear polynomials, thus by $(b)$, $\dim Y\cap Z \geq  n- (2n-r-s) = r+s-n$. 

Note) This is a special case of the projective dimension ( page 48 theorem 1.7.2)

\end{proof}

\begin{problem} \textbf{The d-Uple Embedding} For given $n, d>0$, let
  $M_0,\cdots, M_N$ be all the monomials of degree $d$ in the $n+1$ variables
  $x_0, \cdots x_n$, where $N = \binom{n+d}{n} -1.$ We define a mapping $\rho_d: \PP^n \to \PP^N$ by sending the point $P = (a_0, \cdots, a_n)$ to the point $\rho_d(P) = (M_0(a), \cdots, M_N(a))$ obtained by substituting the $a_i$ in the monomials $M_j$. This is called the $d$-uple \textit{embedding} of $\PP^n$ in $\PP^N$. For example, if $n=1, d=2$, then $N= 2$, and the image $Y$ of the 2-uple embedding of $\PP^1$ in $\PP^2$ is a conic. 
\begin{enuma}

\item Let $\theta: k[y_0, \cdots, y_N] \to k[x_0, \cdots, x_n]$ be the homorphism defined by sending $y_i$ to $M_i$, and let $\mathfrak{a} = \Ker \theta$. Then $\mathfrak{a}$ is homogeneous prime ideal, and so $Z(\mathfrak{a})$ is a projective variety in $\PP^N$. 
\item Show that the image of $\rho_d$ is exactly $Z(\mathfrak{a})$.
\item Now show that $\rho_d$ is a homeomorphism of $\PP^n$ onto the projective variety $Z(\mathfrak{a})$. 
\item Show that the twisted cubic curve in $\PP^3$ is equal to the 3-uple embedding of $\PP^1$ in $\PP^3$, for suitable choice of coordinates. 
\end{enuma}
\end{problem}

\begin{proof}[proof of (a)]
Note that $\theta: y_i \mapsto M_i$, and the degrees of $M_i$'s are all $d$. As all $y_i$'s are mapped to polynomials of the same degree, $\mathfrak{a}= \Ker \theta$ is a homogeneous ideal. To show that $\mathfrak{a}$ is prime, consider $k[y_0, \cdots, y_N]/\mathfrak{a}$. This is isomorphic to a subring of $k[x_0, \cdots, x_n]$. Thus, $k[y_0, \cdots, y_N]/\mathfrak{a}$ is a domain $\iff \mathfrak{a}$ is prime. 
\end{proof}

\begin{proof}[proof of (b)]
Pick $\rho_d(P) \in \Im \rho_d$. Choose any $f \in Z(\mathfrak{a})$. Because $f \in Z(\mathfrak{a}) = \Ker(\theta)$, $f(\rho_d(P)) = f(M_0(a), \cdots, M_N(a)) = 0$. Thus, $\rho_d(P) \in Z(\mathfrak{a})$. Hence, $\Im \rho_d \subseteq  Z(\mathfrak{a})$. 

Conversely, let's show $Z(\mathfrak{a}) \subseteq \Im \rho_d \iff \mathfrak{a} \supseteq I(\Im \rho_d)$. Pick $f \in I(\Im \rho_d)$. Then, $f(Q) = 0$ for all $Q \in \Im \rho_d$. Then, $f(M_0, \cdots, M_N)= 0 \Rightarrow f \in \Ker \theta$. 
\end{proof}

\begin{proof}[proof of (c)]
From $(b)$ we know that $Z(\mathfrak{a}) = \Im \rho_d$. $\rho$ is injective, and the inverse is also continuous.  
\end{proof}

\begin{proof}[proof of (d)] From ex 2.9, we know that the map $\pi: \PP^1 \to \PP^3$ such that $[s:t] \mapsto [s^3: s^2t: st^2: t^3]$ sends $\PP^1$ to the twisted cubic. Choose $M_0(s,t) = s^3, \cdots, M_3 = t^3$. Thus, the twisted cubic curve in $\PP^3$ is equal to the $3$-uple embedding of $\PP^1$ in $\PP^3$. 

\end{proof}


\begin{problem} Let $Y$ be the image of the $2$-uple embedding of $\PP^2$ in $\PP^5$. This is the \textbf{Veronese surface}. If $Z \subseteq Y$ is a closed curve(a curve is a variety of dimension 1), show that there exists a hypersurface $V \subseteq \PP^5$ such that $V \cap Y = Z$. 
\end{problem}

\begin{proof}
The veronese mapping $\nu: \PP^2 \to \PP^5$ is defined to be $\nu: (x:y:z) \mapsto (x^2: y^2:z^2:xy:yz:zx)$. From exercise 2.8, a projective variety $Z$ is a curve (i.e. of dimension 1) if and only if it is the zero set of a single irreducible homogeneous polynomial $f$ of positive degree. Let $Z = Z(f)$ where $f \in k[x,y,z]$ is a homogeneous polynomial of positive degree. Then, $f^2 \in k[x^2:y^2:z^2:xy:yz:zx]$. Thus, there exists a homogeneous polynomial $g \in k[x_0, x_1, \cdots, x_5]$ such that $g|_Y = f$. $V = Z(g)$ is a hypersurface of $\PP^5$, and $V \cap Y = Z$. 
\end{proof}

\begin{problem} \textbf{The Segre Embedding} Let $\psi : \PP^r \times \PP^s \to \PP^N$ be the map defined by sending the ordered pair $(a_0, \cdots, a_r) \times (b_0, \cdots, b_s)$ to $(\cdots, a_ib_j, \cdots)$ in lexicographic order, where $N = rs + r +s$. Note that $\psi$ is well-defined and injective. It is called the \textbf{Segre Embedding}. Show that the image of $\psi$ is a subvariety of $\PP^N$. [Hint: Let the homogeneous coordinates of $\PP^N$ be $\{z_{ij}:i=0, \cdots, r, j = 0,\cdots, s\}$, and let $\mathfrak{a}$ be the kernel of the homomorphism $k[{z_{ij}}] \to k [x_1, \cdots, x_r, y_0, \cdots, y_s]$ which sends $z_{ij}$ to $x_iy_j$. Then show that $\Im \psi = Z(\mathfrak{a})$.]
\end{problem}

\begin{proof}
Let's follow the hint. Let $\theta$ be the homomorphism $\theta: k[{z_{ij}}] \to k [x_1, \cdots, x_r, y_0, \cdots, y_s]$ which sends $z_{ij}$ to $x_iy_j$. If $\mathfrak{a} = \Ker \theta$, Then $z_{ij} z_{kl} \mapsto x_iy_jx_ky_l$, and $z_{kj}z_{il} \mapsto x_ky_jx_iy_l$. Thus, $\mathfrak{a} = (z_{ij}z_{kl} -z_{kj}z_{il}:0\leq i,j \leq r, 0 \leq k,l \leq s)$. We claim that $\Im \psi = Z(\mathfrak{a})$. If $P \in \Im \psi$, $P = (\cdots, a_ib_j,\cdots)$ This for any generator $f = z_{ij}z_{kl}-z_{kj}z_{il}$ of $\mathfrak{a}$, $f(P) = 0 \Rightarrow \Im \psi \subseteq Z(\mathfrak{a})$. Conversely, let's show $Z(\mathfrak{a}) \subseteq \Im\psi \iff \mathfrak{a} \supset I(\Im\psi)$. If $g \in I(\Im \psi)$, then, for every $Q = (\cdots,a_ib_j, \cdots) \in \Im\psi$, $g(Q) = 0$. Then, $g \in \Ker \theta = \mathfrak{a}$. Thus, $\Im \psi = Z(\mathfrak{a})$ is a subvariety of $\PP^N$. 
\end{proof}

\begin{problem} \textbf{The Quadric Surface in $\PP^3$}. Consider the surface $Q$ (a surface is variety of dimension 2) in $\PP^3$ defined by the equation $xy-zw = 0$

\begin{enuma} 
\item Show that $Q$ is equal to the Segre embedding of $\PP^1 \times \PP^1$ in $\PP^3$, for suitable choice of coordinates.
\item Show that $Q$ contains two families of lines (a line is a linear variety of dimension 1) $\{L_t\}, \{M_t\}$, each parametrized by $t \in \PP^1$, with the properties that if $L_t \neq L_u$, then $L_t \cap L_u = \emptyset $; if $M_t \neq M_u$, $M_t \cap M_u = \emptyset$, and for all $t,u, L_t \cap M_u = \mbox{one point}$. 
\item Show that $Q$ contains other curves besides these lines, and deduce that the Zariski topology on $Q$ is not homeomorphic via $\psi$ to the product topology on $\PP^1 \times \PP^1$ (where each $\PP^1$ has its Zariski topology). 
\end{enuma}

\end{problem}

\begin{proof}[proof of (a)] Let $Q = Z(xy -zw) \in \PP^3$. From exercise 2.14, the Segre embedding of $\PP^1 \times \PP^1$ in $\PP^3= \{[z_{00}:z_{01}:z_{10}:z_{11}]\}$ is defined by $Z(z_{00}z_{11}-z_{10}z_{01})$. Thus, after change of coordinates (i.e. $z_{00} \mapsto x, z_{11} \mapsto y, z_{10} \mapsto z, z_{01} \mapsto w$) $Q$  equals to the Segre embedding of $\PP^1 \times \PP^1$ in $\PP^3$. 
\end{proof}

\begin{proof}[proof of (b)] For $t = [u,v] \in \PP^1$, define $L_t$ to be $L_t = \nu (t \times \PP^1)$ where $\nu$ is the Segre embedding $\PP^1 \times \PP^1 \to \PP^3$. Then, $L_t$ is a line(i.e. a linear variety of dimension 1): for $\PP^3 = \{[z_{00}, z_{01}, z_{10}, z_{11}]\}$, $L_t = \{[ux, vy, vx, vy]: [x,y] \in \PP^1 \}= \{[z_{00}, z_{01}, z_{10}, z_{11}]: uz_{00} - vz_{10} = 0\}$. Define $M_u$ similarly, now fixing a point in the latter $\PP^1$. Then, it is immediate that if $L_t \neq L_u$, then $L_t \cap L_u = \emptyset $; if $M_t \neq M_u$, $M_t \cap M_u = \emptyset$, and for all $t,u, L_t \cap M_u = \mbox{one point}$.

\end{proof}



\begin{proof}[proof of (c)] Clearly, $Q$ contains other curves besides these lines: one example would be the curve contained in $Q$ defined by the equation $x = y$. This is closed in $Q$ but not in $\PP^1 \times \PP^1$. Thus, the Zariski topology on $Q$ is not homeomorphic via $\psi$ to the product topology on $\PP^1 \times \PP^1$.

\end{proof}


\begin{problem} \mbox{}
\begin{enuma}
\item The intersection of two varieties need not be a variety. For example, let $Q_1$ and $Q_2$ be the quadric surfaces in $\PP^3$ given by the equation $x^2-yw = 0$ and $xy -zw =0$, respectively. Show that $Q_1 \cap Q_2$ is the union of a twisted cubic curve and a line 
\item Even if the intersection of two varieties is a variety, the ideal of the intersection may not be the sum of the ideals. For example, let $C$ be the conic in $\PP^2$ given by the equation $x^2-yz = 0$. Let $L$ be the line given by $y = 0$. Show that $C \cap L$ consists of one point $P$, but that $I(C) + I(L) \neq I(P)$. 
\end{enuma}
\end{problem}
\begin{proof} [proof of (a)] Let $C =\{st^2:s^2t:s^3:t^3]$ be a twisted cubic (we changed the coordinate), and $L = \{[x,y,z,w]: x = w = 0\}$ be a line in $\PP^3$. We will show that $Q_1 \cap Q_2 = C \cup L$.

First consider when $w = 0$ Then, $x^2 -yw = 0 \iff x^2 = 0$ and $xy - zw =0 \iff xy =0$. Hence, $w = 0 \Rightarrow x = 0$. When, $w \neq 0$, we are in an affine patch $U = U_{w = 1}$, and in $U$, $x^2 -yw = 0$ and $xy -zw = 0$ becomes $x^2 = y, xy = z$. Thus, $U \cap (Q_1 \cap Q_2) = \{[x:x^2: x^3: 1]\}$, which is $C \cap U$. Thus, $C \cup L = (U \cap (Q_1 \cap Q_2)) \cup (U^c \cap (Q_1 \cap Q_2)) = Q_1 \cap Q_2$. Thus, $Q_1 \cap Q_2$ is the union of a twisted cubic curve and a line.

\end{proof}

\begin{proof} [proof of (b)]
If $x^2 -yz = 0, y =0$, then $x^2 =0 \iff x = 0$. Thus, $z = 1$ as $x,y,z$ cannot be all zero. Thus, $C \cap L$ consists of one point $P = [0:0:1]$ Thus, $I(P) = (x,y)$. $I(C) = x^2-yz$ (because $x^2-yz$ is irreducible), $I(L) = y$. Thus, $I(C) + I(L) = (x^2, y)$. Obviously, $(x^2,y) \neq (x,y)$ because $x \in (x,y)$ but $x \not\in (x^2,y)$. 
\end{proof}



\begin{problem} \textbf{Complete intersections.} A variety $Y$ of dimension $r$ in $\PP^n$ is a (strict) complete intersection if $I(Y)$ can be generated by $n-r$ elements. $Y$ is a \textbf{set-theoretic complete intersection} if $Y$ can be written as the intersection of $n-r$ hypersurfaces.
\begin{enuma}
\item Let $Y$ be a variety in $\PP^n$, let $Y = Z(\mathfrak{a})$; and suppose that $\mathfrak{a}$ can be generated by $q$ elements. Then show that $\dim Y \geq n-q$. 
\item Show that a strict complete intersection is a set-theoretic complete intersection.
\item The converse of $(b)$ is false. For example, let $Y$ be the twisted cubic curve in $\PP^3$ (Ex. 2.9). Show that $I(Y)$ cannot be generated by two elements. On the other hand, find hypersurfaces $H_1, H_2$ of degrees 2, 3 respectively, such that $Y = H_1 \cap H_2$. 
\item It is an unsolved problem whether every closed irreducible curve in $\PP^3$ a set-theoretic intersection of two surfaces.
\end{enuma}

\begin{proof} [proof of (a)] Let $Y = Z(\mathfrak{a})$ be a variety in $\PP^n$ where $\mathfrak{a}$ can be generated by $q$ elements. Use induction on $q$ to show that $\dim Y \geq n-q$. If $q = 1$, the result follows from Exercise 2.8. 

Assume that the result is true for $q = r$. Now, $\mathfrak{a} = (e_1, \cdots, e_r, e_{r+1})$ where $e_{r+1} \not\in (e_1, \cdots, e_r)$. Let $Y' = Z(e_1, \cdots, e_r)$ and by induction hypothesis, $\dim Y' \geq n-r$. From Exercise 1.8, $\dim Y = \dim (Y' \cap Z(e_{r+1})) \geq n-r-1 = n-(r+1)$. 
\end{proof}

\begin{proof} [proof of (b)] Assume that a $r$-dimensional variety $Y \subset \PP^n$ is a strict complete intersection i.e. $I(Y)$ can be generated by $n-r$ elements. Let $I(Y) = (f_1, \cdots, f_{n-r})$.  Then, $Y = \cap_i Z(f_i)$: $x \in Y \iff f_i(x) = 0 $ for all $i = 1, \cdots, n-r \iff$ $x \in \cap Z(f_i)$. Hence, $Y$ is a set-theoretic complete intersection. 

\end{proof}

\begin{proof} [proof of (c)] Now, let $Y = \{[s^3:s^2t:st^2:t^3]\}$ be the twisted cubic curve in $\PP^3$. Then, $Y = Z(xz-y^2, yw-z^2, xw-yz)$, and $(xz-y^2, yw-z^2, xw-yz)$ is a radical ideal. $I(Y)=(xz-y^2, yw-z^2, xw-yz)$ cannot be generated by any of the two polynomials $f,g \in k[x,y,z,w]$: if so, $xw-yz \in (xz-y^2, yw-z^2)$ but it is easy to check that this is not true (exercise 1.11).

However, we can find hypersurfaces $H_1, H_2$ of degrees 2, 3 respectively, such that $Y = H_1 \cap H_2$. Let $H_1 = Z(z^2-yw)$ and $H_2 = Z(y^3-2xyz+x^2w)$. Then, $H_1 \cap H_2 = Z(\sqrt{(z^2-yw, y^3-2xyz+x^2w)})$. We can easily check (by Macaulay or easily by hand) that $\sqrt{(z^2-yw, y^3-2xyz+x^2w)} = (xz-y^2, yw-z^2, xw-yz)$. Hence, $H_1 \cap H_2 = Y$. Hence the converse of $(b)$ is false. 

\end{proof}

\begin{proof} [proof of (d)] 

Wingardium Leviosa! Avada Kedavra! Algebraic Geometry! Grothendieck! Robin Hartshorne!
\end{proof}



\end{problem}


\printbibliography
\end{document}
